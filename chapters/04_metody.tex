\section{Mise DIANA}
\label{sec:mise_diana}
Tato diplomová práce těží z již druhé vesmírné analogové mise, která simulovala
přistání na měsíci a uskutečnila se v rámci projektu Hydronaut v létě roku 2022.
Jednotlivé kompartmenty mise měly následující role: řídící věž byla stanoviště
na Zemi, mateřská loď obíhala na oběžné dráze Měsíce a přistávací modul byl na
povrchu Měsíce. Blíže jsou dílčí kompartmenty popsány v následujících sekcích.

Mise primárně sloužila pro zkoumání vlivu osobnostních charakteristik a vnějších
faktorů na dynamiku týmu při dlouhodobém pobytu v \gls{ICE} prostředí (projekt
TAČR ÉTA č. TL05000228). Mise DIANA byla podpořena Evropskou kosmickou
agenturou, vzhledem k jejímu potenciálu pro výcvik astronautů simulací
extrémního prostředí. 
\subsection{Mise Diana}
\label{subsec:mise_diana}

\subsection{Měření biosignálů}
\label{subsec:_mereni_biosignalu}



\section{Použité datasety}
\label{sec:datasety}
Pro účely diplomové práce bylo použito několik veřejně dostupných datasetů
včetně dat z mise DIANA. V následujících sekcích jsou stručně rozebrány
jednotlivé datasety.
\subsection{WESAD}
\label{subsec:wesad}
Dataset WESAD~\cite{wesadDataset} (\textit{Wearable Stress and Affect
Detection}) je multimodální dataset navržený pro výzkum v oblasti hodnocení
stresu a emocí za použití nositelných senzorů. Tento dataset byl vytvořen s
cílem přispět k vývoji pokročilých algoritmů strojového učení pro analýzu
fyziologických signálů a rozpoznání emocí. WESAD obsahuje data získaná od 15
účastníků, přičemž každý z nich prošel sérií experimentů v laboratorních
podmínkách.

Data byla v datasetu získána ze dvou nositelných zařízení. Prvním zařízením byl
\textit{RespiBAN}\footnote{Zařízení \textit{RespiBAN} již není vyráběno},
nositelný senzor umístěný na hrudi se vzorkovací frekvencí 700~Hz, který
zaznamenával elektrokardiogram, elektrodermální aktivitu, elektromyogram,
respirační signál a teplotu těla. Druhé zařízení byl chytrý náramek
\textit{Empatica E4}\footnote{\url{https://www.empatica.com/research/e4}}, který
zaznamenává krevní tlak (64~Hz), elektrodermální aktivitu (4~Hz), teplotu těla
(4~Hz) a tříosou akceleraci (32~Hz). Experiment, který byl proveden pro sběr
dat, zahrnoval celkem čtyři fáze:
\begin{enumerate}
    \item \textbf{Základní úroveň (Baseline condition)} --- účastník byl požádán, aby
    seděl/stál v klidu po dobu 20 minut.
    \item  \textbf{Stresový úkol (Stress condition)} --- účastník musel pět minut
    přednášet před publikem a poté vyřešit matematický úkol, který byl navržen
    tak, aby vyvolal stres.
    \item  \textbf{Relaxační úkol (Amusement condition)} --- účastník sledoval
    komediální video, které mělo vyvolat příjemné emoce.
    \item  \textbf{Řízená meditace (Meditation)} --- účastník prováděl řízenou
    meditaci, jejíž cílem bylo navození do stavu blízkého neutrálnímu
    afektivnímu stavu.
\end{enumerate}

WESAD poskytuje časově synchronizovaná, předzpracovaná a anotovaná data z těchto
nositelných senzorů. Pro účely diplomové práce byly použity signály ze zařízení
\textit{RespiBAN}, konkrétně srdeční, respirační a elektrodermální aktivita. Pro
úlohy hodnocení kognitivní zátěže byla základní úroveň označena jako klidový stav
a stresové úkoly byly označeny jako stav kognitivní zátěže.

\subsection{CLAS}
\label{subsec:clas}
Dataset CLAS~\cite{clasDataset} (\textit{Cognitive Load, Affect, and Stress
Recognition}) je podobně jako WESAD multimodální dataset vytvořený pro výzkum v
oblasti rozpoznávání kognitivní zátěže a emocí za použití nositelných senzorů a
dalších datových zdrojů. Dataset zahrnuje data získaná od 62 účastníků, kteří
byli vystaveni různým úkolům a podnětům navrženým tak, aby vyvolaly různé úrovně
kognitivní zátěže a emocí. Mezi tyto podněty patřily například matematické úlohy
nebo Stroopův test. Každý účastník byl zároveň měřen i v klidu (dataset stav
označuje jako Baseline). Během přechodů mezi jednotlivými úkoly bylo účastníkům
puštěno neutrální video nebo byl účastník požádán, aby vyplnil dotazník (dataset
označuje jako Neutral). 

Fyziologická data v rámci tohoto datasetu byla měřena zařízením
\textit{Shimmer3}\footnote{\url{https://shimmersensing.com}} se vzorkovací
frekvencí 256~Hz. Mezi měřené biologické signály patří elektrokardiogram,
elektrodermální aktivita a fotopletysmogram. Pro úlohy hodnocení kognitivní
zátěže byl Baseline stav označen jako klidový stav a stresové úkoly byly
označeny jako stav kognitivní zátěže. 

\subsection{Data z vesmírné analogové mise DIANA}
\label{subsec:data_diana}
Pro potřeby diplomové práce poskytla \gls{FF UPOL} data z vesmírné analogové
mise DIANA. Součástí dat jsou osmidenní 24 hodinové záznamy biologických
signálů, kamerových záznamů a anotace v podobě časů kognitivních úloh pro
každého člena posádek. Využitými signály v této práci jsou elektrokardiogram
spolu s elektrodermální a respirační aktivitou.

\section{Zpracování biosignálů}
\label{sec:zpracovani_biosignalu}
Důležitým krokem při analýze biosignálů je jejich zpracování. V této sekci jsou
popsány použité algoritmy, které byly implementovány v programovacím jazyce
Python, využitím knihovny \textit{Neurokit2} (viz sekce~\ref{subsec:neurokit}).
\subsection{Zpracování elektrické srdeční aktivity}
\label{subsec:zpracovani_ekg}
Pro zpracování EKG záznamů byla implementována metoda podle Kalidase a
Tamila~\cite{kalidas2017}, která je založena na stacionární vlnkové transformaci
(\gls{SWT}). Metoda vychází z populárního Pan-Tompkinsova~\cite{Tompkins1985}
algoritmu ale pro odstranění šumu a zvýraznění QRS komplexů používá \gls{SWT}
namísto pásmové propusti. Stacionární vlnková transformace je metoda rozkladu
signálu na jednotlivá frekvenční pásma pomocí mateřské vlnky~\cite{Nason1995}.
Metoda byla zvolena na základě následující sekce~\ref{subsubsec:vyberqrs}.

\begin{figure}[H]
    \begin{center}
        \includegraphics[width=1\linewidth]{figures/kalidas2017}
        \caption{\textbf{A-D)} Kroky zpracování EKG pro algoritmus podle
            Kalidase a Tamila~\cite{kalidas2017} \textbf{E)} Frekvenční spektrum
            nefiltrovaného EKG se vzorkovací frekvencí 250~\si\Hz~(modrá) a EKG po
            SWT 3. řádu (oranžová). \textbf{F)} vlnka rodiny Daubechies 3. řádu.
            (Upraveno a převzato z~\cite{Porr2019})}
        \label{fig:kalidas_processing}
    \end{center}
\end{figure}

V tomto algoritmu se stacionární vlnková transformace provádí na EKG signálu
využitím Daubechiesové vlnky třetího řádu. Po provedení \gls{SWT} se extrahují
koeficienty, které se následně vyčíslí na čtverec. Následně je využito filtrace
pásmovou propustí ke zvýšení citlivosti a přesnosti detekce. Postup detekce R
vln na filtrovaném signálu je pak totožná s detekcí Pan-Tompkinsova
algoritmu~\cite{Tompkins1985}. Jednotlivé částí zpracování lze vidět na
Obr.~\ref{fig:kalidas_processing}.

\subsubsection{Metodika výběru QRS detektoru}
\label{subsubsec:vyberqrs}
Vzhledem k tomu, že neexistuje žádný jednotný standard či systematický postup
pro zpracování EKG a výběr \enquote{správného} QRS detektoru pro určitou
aplikaci, tak bylo v rámci této práce realizováno statistické porovnání
populárních algoritmů. Jakožto měřítko přesnosti QRS detekce algoritmu bylo
vycházeno z výpočtu absolutní vzdálenosti od původní \enquote{skutečné} polohy R
vlny. Pro benchmarking detektorů byly tedy použity následující anotované
datasety:

\begin{table}[h]
    % \footnotesize
    \begin{center}
        \caption{\label{tab:bench_datasets} Vybrané datasety pro benchmarking
            QRS detektorů z PhysioNetu~\cite{PhysioNet}}
        \renewcommand{\arraystretch}{1.3}
        \begin{tabular}{p{12cm}c}
            \toprule
            \textbf{Dataset}                                                                                                 & \textbf{Probandi} \\ \midrule
            MIT-BIH Arrhythmia Database~\cite{MITBIHArrhythmia}                                                              & 48                \\
            MIT-BIH Normal Sinus Rhythm Database~\cite{Beth1990}                                                             & 18                \\
            Glasgow University Database~\cite{GUDB}                                                                          & 25                \\
            Fantasia Database~\cite{FANTASIA}                                                                                & 40                \\
            Lobachevsky University Electrocardiography Database~\cite{LUDB}                                                  & 200               \\
            Simultaneous physiological measurements with five devices at different cognitive and physical loads~\cite{IFADO} & 13                \\
            Pulse Transit Time PPG Dataset~\cite{USYD}                                                                       & 22                \\
            \bottomrule
        \end{tabular}
    \end{center}
\end{table}

Byly vybrány různorodé datasety za účelem zjištění adaptability algoritmu. Pro
statistické zpracování bylo využito lineárních smíšených modelů (\gls{LMM}).
Použité statistické metody jsou podrobněji popsány v
kapitole~\ref{sec:statisticke_metody}. Pro srovnání metod byl v programovacím
jazyce R vytvořen následující statistický model:
\begin{equation}
    \text{Skóre} = \beta_0 + \beta_1\text{Metoda} + u_{\text{Dataset}} + u_{\text{Participant}} + \epsilon
\end{equation}
který specifikuje lineární smíšený model pomocí funkce \texttt{lmer} z balíku
\texttt{lme4}\footnote{\url{https://github.com/lme4/lme4}}. Model byl použit k
predikci závislé proměnné Skóre na fixním efektu $Metoda$ a dvou náhodných
efektech: $Dataset$ a $Participant$. Dále problematice této sekce není věnována
pozornost, jelikož není předmětem této práce. 

\subsection{Zpracování respirační aktivity}
\label{subsec:zpracovani_rsp}
Pro zpracování respirační aktivity byl implementován
algoritmus~\cite{Khodadad2018}, který je založen na průchodech nulou (\gls{ZC},
Zero-Crossing). Originální signál je nejdříve filtrován pásmovou propustí pro
odstranění stejnosměrné složky, aby bylo možné spolehlivě detekovat \gls{ZC}. K
tomu byla využita pásmová propust 0,05--3~Hz , která zároveň zachovává dechové
frekvence menší než tři a vyšší než 180 dechů za minutu. Následně jsou pomocí
logických operací detekovány indexy náběžných a sestupných průchodů nulou, mezi
kterými došlo k hledání lokálních extrémů.

\begin{figure}[h]
    \begin{center}
        \includegraphics[width=1\linewidth]{figures/rsp_test}
        \caption{Příklad zpracování RSP pomocí implementované metody}
        \label{fig:rsp_test}
    \end{center}
\end{figure}

Zachovány jsou ve výsledku pouze ty extrémy, které mají minimální vertikální
vzdálenost od svého přímého souseda, tudíž kritérium pro detekci odlehlých
hodnot bylo definováno v absolutním rozdílu amplitud mezi sousedními extrémy.
Aplikace metody na reálném signálu lze vidět na Obrázku~\ref{fig:rsp_test}.

\subsection{Zpracování elektrodermální aktivity}
\label{subsec:zpracovani_eda}
Zpracování elektrodermální aktivity vychází z metod~\cite{vanhalem2020}
a~\cite{posada2016}. Signál je nejdříve filtrován Butterworthovou horní propustí
4. řádu s mezní frekvencí 3~\si\Hz. Následně je ze signálu extrahována fázická a
tonická složka pomocí dolní a horní propusti o mezních frekvencích 0,05~\si\Hz.

\begin{figure}[h]
    \begin{center}
        \includegraphics[width=1\linewidth]{figures/eda_test}
        \caption{Příklad zpracování EDA pomocí implementovaných metod}
        \label{fig:eda_test}
    \end{center}
\end{figure}

Dále byl použit Savitzky-Golayův frekvenčně neselektivní filtr k dalšímu
vyhlazení fázické složky za účelem hledání \gls{SCR} vrcholků. K detekti
vrcholků byla využita funkce \texttt{find\_peaks()} z knihovny
Scipy\footnote{\url{https://docs.scipy.org}}. Kritérium pro detekci vrcholků
bylo definováno jako konzistentní nárůst o 0,5s následovaný stejným poklesem.
Výsledek aplikované metody lze vidět na Obrázku~\ref{fig:eda_test}.

\subsection*{Neurokit}
\label{subsec:neurokit}
Knihovna \textit{Neurokit2}\footnote{\url{https://neuropsychology.github.io/NeuroKit}},
na jejíž vývoji se podílím, poskytuje pokročilé metody pro zpracování a
vizualizaci biosignálu. Jednotlivé metody zároveň nabízejí možnost si vybrat z
mnoha implementovaných algoritmů. V této práci byla knihovna použita pro
zpracování respirační, elektrodermální a elektrické srdeční aktivity včetně
zpracování a výpočet \gls{HRV} parametrů.


\section{Zpracování dat z mise DIANA}
\label{sec:zpracovani_dat_diana}
\subsection{Zpracování exportovaných segmentů biosignálů}
\label{subsec:prezpracovani_segmentu}
Pro každého člena posádky byly exportovány segmenty biosignálů o délce 30s s
50\% překryvem na základě poznatků
v~\cite{Castaldo2019,Kim2021,Pecchia2018,Shaffer2020,Tervonen2021}. Zpracování
biosignálu vycházelo z metodiky popsané v sekci~\ref{sec:zpracovani_biosignalu}.
U každého segmentu proběhlo hodnocení jeho kvality podle dvou kritérií:
\begin{itemize}
    \item Hodnocení kvality \gls{EKG} signálu pomocí heuristické fúze a fuzzy
    komplexního hodnocení podle~\cite{Zhao2018}.
    \item Hodnocení detekovaných R vln z hlediska časové kontroly náhlých
    nefyziologických změn v po sobě jdoucích R-R intervalech.
\end{itemize}
Segmenty které vykazovali nežádoucí anomálie v rámci hodnotících kritérií byly
vyřazeny. Ze segmentů bylo dále vypočteno následně přes 100 různých parametrů
pro účely analýzy dat. Mezi tyto parametry patřily například běžné statistické
charakteristiky (průměr, medián, směrodatná odchylka a další) nebo nelineární a
časové \gls{HRV} parametry. Zpracované segmenty byly zároveň anotovány, a to z
hlediska spánkového cyklu. Dále byly identifikovány a označeny v časech
kognitivních testů. Z časových důvodu nebyly pro účely této práce ostatní
aktivity během mise anotovány, i přes dostupnost kamerových záznamů. Seznam
všech počítaných parametrů je součástí přílohy v souboru
\texttt{all\_params.csv}.

\subsection{Čistění dat}
\label{subsec:cisteni_dat}
Ze souborů vypočtených parametrů byly vynechány všechny parametry, jejichž
sloupce obsahovali \texttt{NaN} hodnoty. Dále byly parametry korelovány a
odstranili se ty, které byly vzájemně dokonale korelované ($|r| > 0,999$). Poté
se odstranili odlehlé hodnoty na základě absolutní odchylky mediánu od mediánu.

\subsection{Sledované veličiny}
\label{subsec:sledovane_veliciny}
Pro účely analýzy \gls{NPF} adaptace probandů v průběhu mise byly vybrány a
sledovány především následující \gls{HRV} parametry:

\subsection{Tvorba hypotetických LMM modelů}
\label{subsec:tvorba_modelů}



\section{Explorační analýza dat}
\label{sec:exploracni_analyza}
\subsection{Předzpracování datasetů}
\label{subsec:predzpracovani_datasetu}
Z datasetů byly extrahovány potřebné biosignály, které byly následně zpracovány
a normalizovány na úrovní subjektů. Metodika zpracování vychází ze
sekce~\ref{sec:zpracovani_biosignalu}. Normalizace byla provedena škálováním
biosignálů tak, aby měly průměrnou hodnotu nula a směrodatnou odchylku jedna. To
bylo dosaženo odečtením střední hodnoty biosignálu od každé jeho hodnoty a
následným vydělením směrodatnou odchylkou.

Vzhledem k tomu, že dataset CLAS neobsahuje signál RSP, byl tento signál
vytvořen pomocí metody \gls{EDR} (ECG-Derived Respiration). Jedná se o extrakci
informace o dýchání z elektrokardiogramu. Knihovna \textit{Neurokit2} poskytuje
implementaci algoritmu podle~\cite{VanGent2019}, jež byla pro tyto účely
použita.

Následně byly všechny signály segmentovány na 5s, 5s s 50\% překryvem a 1s
úseky. Z těchto segmentů byly poté vytvořeny příznaky pro účely strojového
učení. Blíže je tvorba těchto příznaků a jejich využití popsáno v
sekci~\ref{sec:hybridni_detekce}. Byly ponechány pouze ty segmenty, které svojí
třídou korespondovaly žádanému kognitivnímu stavu.

\begin{figure}[h]
    \centering
    \framebox[\textwidth]{%
        \begin{subfigure}[b]{0.45\textwidth}
            \dirtree{%
                .1 Datasets.
                .2 WESAD.
                .3 S2.
                .4 S2.pkl.
                .4 $\vdots$.
                .3 S3.
                .3 $\vdots$.
                .2 CLAS.
                .3 Part1.
                .4 full\_ecg.csv.
                .4 full\_gsr\_ppg.csv.
                .4 $\vdots$.
                .3 Part2.
                .3 $\vdots$.
            }
            \caption{Původní struktura datasetů}
            \label{subfig:tree1}
        \end{subfigure}
        \hfill
        \begin{subfigure}[b]{0.45\textwidth}
            \dirtree{%
                .1 Datasets.
                .2 WESAD.
                .3 merged\_1s.pkl.
                .3 merged\_5s.pkl.
                .3 merged\_5s\_2s.pkl.
                .3 $\vdots$.
                .2 CLAS.
                .3 merged\_1s.pkl.
                .3 merged\_5s.pkl.
                .3 merged\_5s\_2s.pkl.
                .3 $\vdots$.
            }
            \caption{Zpracované datasety}
            \label{subfig:tree2}
        \end{subfigure}
    }
    \caption{Porovnání struktury původních a zpracovaných datasetů}
    \label{fig:struktura_datasetu}
\end{figure}

\subsection{Explorace dat}
\label{subsec:explorace_dat}
Oba datasety byly po předzpracování zkoumány pomocí nástrojů knihovny
\textit{Pandas} v programovacích jazyce Python. Datasety tak byly popsány
například základními statistickými údaji, kde byla brána primárně zřetel na
rozdělení tříd. Na Obr.~\ref{fig:rozdeleni_trid} lze tak vidět, že oba datasety
vykazují výraznou nevyváženost tříd. Dále byly korelovány jednotlivé signály
datasetů, bylo nahlíženo na jejich distribuce, a byly vyšetřeny případné záporné
nebo neplatné hodnot. Tato šetření byla provedena i na individuální úrovni. Celý
postup s vizualizovanými výsledky se nachází v datové příloze, v podobě
interaktivního prostředí Jupyter Notebook\footnote{\url{https://jupyter.org}} s
názvem souboru \texttt{exploratory\_analysis\_example}.

\begin{figure}[h]
    \begin{subfigure}[h]{0.48\linewidth}
        \includegraphics[width=\linewidth]{figures/wesad_labels}
        \caption{Rozdělení datasetu WESAD}
    \end{subfigure}
    \hfill
    \begin{subfigure}[h]{0.48\linewidth}
        \includegraphics[width=\linewidth]{figures/clas_labels}
        \caption{Rozdělení datasetu CLAS}
    \end{subfigure}
    \caption{Srovnání rozdělení tříd vybraných datasetů. Třída 0 vyjadřuje
    klidový stav a třída 1 vyjadřuje kognitivní zátěž.}
    \label{fig:rozdeleni_trid}
\end{figure}

Během explorace dat byla také zkoumána separovatelnost dat v
nízkodimenzionálních prostorech použitím techniky \gls{UMAP}~\cite{umap2018}
(Uniform Manifold Approximation and Projection) právě pro redukci
dimenzionality. Cílem bylo vizualizovat soubor dat ve formě nižších dimenzí k
získání přehledu o rozdělení tříd a nadhled nad smyslem a rozpoložení dat.
Výsledky této vizualizace je možné vidět na Obr.~\ref{fig:umap}, kde jsou
oranžově vyznačeny body, jež odpovídají kognitivní zátěži.

\begin{figure}[h]
    \begin{subfigure}[h]{0.48\linewidth}
        \includegraphics[width=\linewidth]{figures/wesad_umap}
        \caption{WESAD}
    \end{subfigure}
    \hfill
    \begin{subfigure}[h]{0.48\linewidth}
        \includegraphics[width=\linewidth]{figures/clas_umap}
        \caption{CLAS}
    \end{subfigure}
    \caption{Vizualizace UMAP projekcí extrahovaných příznaků z datasetu WESAD
    do 2D (vlevo) a 3D latentního prostoru (vpravo). Oranžově třída 1 a modře
    třída 0}
    \label{fig:umap}
\end{figure}

V metodě bylo využito Euklidovské vzdálenostní metriky s hodnotou 0,1 a počtem
sousedních bodů 15. Tyto nízké hodnoty byly zvoleny pro účely zachycení lokální
struktury dat (potenciálně na úkor celkového obrazu), jak ukazuje
Obr.~\ref{fig:umap}. Výsledek 3D projekcí naznačuje potencionální lineární
separovatelnost u malé části souboru. U zbytku souboru by bylo oddělení
pravděpodobně zřetelnější ve vyšších dimenzích a možná nebude lineární. V úvahu
tak přichází strojové učení.

\section{Konvenční parametrická detekce CL}
\label{sec:kovencni_detekce}
\input{chapters/metody/konvencni_detekce}

\section{Detekce CL využitím vícerozměrných časoprostorových kauzálních vzorů}
\label{sec:hybridni_detekce}
\subsection{Definice problému}
\label{subsec:definice_problemu}
V předešlé sekci byl popsán proces předzpracování dat, jehož výsledkem je nová
multidimenzionální množina segmentovaných fyziologických signálů $\mathcal{F}$.
Dimenze této množiny je rovna počtu snímaných signálů, lze tedy dál hovořit jako
o kanálech $c$. Vybraný segment z kanálu $c$ reprezentuje časovou řadu, jejiž
každý vzorek je zachycen v určitém čase $t$, a představuje průběh fyziologické
události (\gls{NPF} událost).

Mezi jednotlivými pozorovanými fyziologickými událostmi $X^i$, lze modelovat
časově kauzální vztahy, které lze díky Grangerově kauzalitě (viz
sekce~\ref{subsec:granger}) matematicky popsat unikátním orientovaným grafem --
definovanou množinou vrcholů a hran. Tím je umožněno temporální kódování
specifického příčinného kognitivního stavu pro konkrétní segment vybraného
kanálu $c$. Nelze však předpokládat, že je tak zachycena veškerá komplexní
dynamika, která je ve biosignálech přítomna, včetně toho, že nemusí být plně
zachyceny interakce napříč kanály $c$.

Tento problém je kompenzován použitím vícerozměrných časoprostorových vzorů
(\gls{GAF}, Gramian Angular Fields) odvozených z Gramových matic. Tyto vzory
zachycují určitý druh temporální i prostorové korelace v rámci fyziologických
událostí. V následujících sekcích je dále popsána tvorba zmíněných příznaků
společně s jejich aplikací v rámci strojového učení.

\subsection{Tvorba kauzálních vícerozměrných matic}
\label{subsec:kauzalni_matice}
Pro zachycení časově kauzálních relací v použitých biosignálech byl zvolen
přístup Kopula-Granger s Lasso ($\ell_1$) regularizací, který kombinuje koncept
Grangerovy kauzality s teorií kopulí. Zmíněné přístupy spadají do oblasti
statistických metod, a proto jsou dále jednotlivé popsány v
sekci~\ref{sec:statisticke_metody}.

Vytvoření kauzální matice ja založeno na použití fyziologické události $X^i$
definované v minulé sekci, ze které lze uplatněním Kopula-Granger metody získat
robustní odhad koeficientů vektorů $\beta_i$ pro test Grangerovy kauzality
využitím regresní úlohy. K tomu je ale potřeba vyřešit následující optimalizační
problém~\cite{Schindler2013,Guy2016}:
\begin{equation}
    \min _{\beta_i} \sum_{l=L+1}^T\left|X_t^i-\sum_{j=1}^p\left(X_{t, \text {lag}}^j\right) \cdot\left(\beta_i^j\right)^{\prime}\right|^2+\lambda\left\|\beta_i\right\|_1,
\end{equation}
kde $\lambda$ je penalizační parametr ovlivňující řídkost vektoru $\beta_i$, $L$
je maximální časové zpoždění (lag) a ${X}_{t, \text {lag}}^j$ jsou předchozí
hodnoty řady $X^j$ v čase $[t - L, t - 1]$. Podle definice Kopula-Granger
modelu~\cite{Schindler2013,Guy2016} lze dále uplatnit faktorizaci na základě
modelu vektorové autoregrese (\gls{VAR}) s koeficienty $B = {{\beta}_{i}^j}$
následovně:
\begin{equation}
    p_Z(z)=\mathcal{N}(z(1, \ldots, L)) \times \prod_{j=1}^n \prod_{t=L+1}^T p_{\mathcal{N}}\left(z_j(t) ; \sum_{t=1}^n \beta_{i, j}^T z_i^{t, \text {lag}}, \sigma_j\right)
\end{equation}
kde $p_{\mathcal{N}}(z ; \mu, \sigma)$ je Gaussova funkce se střední hodnotou
$\mu$ a rozptylem $\sigma^2$, $z_i^{t, \text {lag}}$ jsou předchozí hodnoty
$z_i$ do času $t$ a ${\beta}_{i}^j$ je vektor koeficientů modelujících vliv
časové řady $z_j$ na cílovou časovou řadu. Kauzalita je tedy definována časovou
řadou $z_j$, jež je příčinou $z_i$ pokud je alespoň jedna hodnota vektoru
${\beta}_{i}^j$ nenulová ve smyslu statistické významnosti
(viz~\ref{subsec:granger}).

\begin{figure}[h]
    \begin{center}
        \includegraphics[width=1\linewidth]{figures/GCN}
        \caption{Diagram tvorby kauzálních vícerozměrných matic. 1) Aplikace
            Kopula-Granger metody pro vybraný segment všech kanálů $c$. 2) Kombinace
            výsledných kauzálních matic do jednoho trojrozměrného pole. 3) Výsledný
            příznak ve smyslu RGB obrázku}
        \label{fig:GCN}
    \end{center}
\end{figure}

Kopula-Granger metodu lze primárně shrnout do dvou kroků: odhadnutí marginální
distribuční funkce vybrané časové řady $X^i$ jako $\hat{F_i}$ a mapování
pozorovaných hodnot fyziologické události v čase $t$ do kopula prostoru jako
$Z_{i}^t=\Phi^{-1}\left(\hat{F}_i\left(X_{t}^i\right)\right)$, kde $\Phi$ je
kumulativní distribuční funkce (\gls{CDF}) Gaussova rozdělení. V neposlední řadě
lze konstruovat temporální kauzální graf analýzou relací mezi $Z_{i}^t$. Pro
účely tvorby grafického řešení v podobě kauzální matice $T \times T$ obsahující
odhady koeficientů $\hat{\beta}_i$ byla adaptována regularizovaná regrese
podle~\cite{Bahdori2012}:
\begin{equation}
    \hat{\beta}_i(\lambda)=\arg \min _{\beta_i}\left(\sum_{t=1}^T\left\|x_i^t-X_{t, L}^{\text {lag}} \beta_i\right\|^2+\lambda\left\|\beta_i\right\|_1\right)
\end{equation}
kde $x_i^t$ představuje pozorovanou hodnotu v časové řadě a $X_{t, L}^{\text {lag}}$
reprezentuje spojený vektor všech zpožděných pozorování. Tvorba
kauzálních matic byla realizována v programovém prostředí Matlab s využitím
knihovny \textit{GLMNET}\footnote{\url{https://hastie.su.domains/glmnet_matlab}},
která umožňuje specifikovat různé typy regresních modelů. Hodnota časového
zpoždění byla určena na základě Akaikeho informačního kritéria (\gls{AIC}) jako
$L = 4$. Během každého výpočtu bylo pomocí zmíněné knihovny realizováno i
automatické ladění hodnot penalizačního parametru $\lambda \in m_i$, kde $m_i =
    10^{a + (i-1)d}$ pro $i = {1, 2, ..., 6}$ a $d = \frac{b-a}{n-1}$. Hodnoty $a$ a
$b$ byly nastaveny na -3 a 2.

\subsection{Konstrukce časoprostorových polí}
\label{subsec:gadf}
Bylo realizováno mapování segmentů všech kanálů $c$ do prostorové domény
(\gls{GAF}), jež zachovává časové závislosti. Wang and Oates~\cite{Wang2015}
představili koncept této metody transformace časové řady do 2D obrazu v roce
2015. Vzhledem k dříve definované fyziologické události, tedy vybrané časové
řadě $X^i = \{x_1^i, x_2^i, \dots, x_t^i\}$ z kanálu $c$ zahrnuje konstrukce
\gls{GAF} nejdříve normalizaci na interval $[-1; 1]$:
\begin{equation}
    \tilde{x}_t=\frac{\left(x_t-\max (X^i)+\left(x_t-\min (X^i)\right)\right.}{\max (X^i)-\min (X^i)}
\end{equation}

\begin{figure}[h]
    \begin{center}
        \includegraphics[width=1\linewidth]{figures/polar}
        \caption{Ukázka \gls{GAF} mapování na EKG segmentu~(Upraveno a převzato z~\cite{Zhou2021})}
        \label{fig:polar}
    \end{center}
\end{figure}

Dále jsou normalizovaná data časové řady převedeny do polárních souřadnic
výpočtem úhlové složky $\theta_i$ a radiální složky $r_i$ pro každý vzorek
$\tilde{x}_t$:
\begin{equation}
    \begin{cases}
        \theta_i = \arccos(\tilde{x}_t), & -1 \leq \tilde{x}_t \leq 1, \tilde{x}_t \in \tilde{X^i} \\
        r_i = \frac{t}{N},               & t \in N
    \end{cases}
\end{equation}
kde $t$ je časová značka vzorku fyziologické události a $N$ je konstantní faktor
pro regulaci rozpětí polárního souřadného systému. Jinými slovy, časová značka
představuje poloměr a arkus kosinus hodnoty časové řady úhel. Lze zde hovořit o
bijektivní transformaci, jež zachovává časovou závislost pomocí souřadnice $r$.

Po transformaci přeškálované časové řady do polárního souřadnicového systému lze
využít úhlovou perspektivu, v tomto případě s ohledem na trigonometrický rozdíl
mezi jednotlivými body (\gls{GADF}, Gramian angular field difference), k
identifikaci temporální korelace v rámci různých časových intervalů:
\begin{equation}
    GADF = \left[\sin \left(\phi_i-\phi_j\right)\right]
\end{equation}

\begin{figure}[h]
    \begin{center}
        \includegraphics[width=1\linewidth]{figures/GADF}
        \caption{Diagram tvorby časoprostorových vzorů. 1) Aplikace GAF mapování
            pro vybraný segment všech kanálů $c$. 2) Kombinace výsledných polí
            do jednoho trojrozměrného pole. 3) Výsledný příznak kódující
            temporální korelace fyziologické události v prostorové doméně}
        \label{fig:gadf}
    \end{center}
\end{figure}

Ve výsledku je tedy časoprostorový vzor definován následující $T \times T$
maticí, která je kvazi-Gramovou maticí:
\begin{equation}
    GADF = \left[\begin{array}{cccc}
            \sin \left(\phi_1-\phi_1\right) & \sin \left(\phi_1-\phi_2\right) & \cdots & \sin \left(\phi_1-\phi_n\right) \\
            \sin \left(\phi_2-\phi_1\right) & \sin \left(\phi_2-\phi_2\right) & \cdots & \sin \left(\phi_2-\phi_n\right) \\
            \vdots                          & \vdots                          & \ddots & \vdots                          \\
            \sin \left(\phi_n-\phi_1\right) & \sin \left(\phi_n-\phi_2\right) & \cdots & \sin \left(\phi_n-\phi_n\right)
        \end{array}\right]
\end{equation}
kde každý prvek odpovídá sinové funkci úhlového sinusového rozdílu v různých
časových bodech. Výpočet a tvorba těchto příznaků, časoprostorových vzorů, byla
implementována v programovacím jazyce Python.

\subsection{Augmentace dat}
\label{subsec:augmentace_dat}

\subsection{Kapsulární neuronová síť}
\label{subsec:kapsularni_sit}
Pro potřeby realizace úloh klasifikace (resp. detekce kognitivní zátěže), byla
navržena architektura kapsulární neuronové sítě postavená na řešení, které
představili Mazzia et al.~\cite{Mazzia2021}, \textit{Efficient-CapsNet}.
Celkovou architekturu lze vidět na obrázku~\ref{fig:architektura}.

\begin{figure}[h]
    \begin{center}
        \includegraphics[width=1\linewidth]{figures/capsule}
        \caption{Schematické znázornění architektury sítě
            \textit{Efficient-CapsNet} (Upraveno a převzato z~\cite{Zhou2021})}
        \label{fig:architektura}
    \end{center}
\end{figure}

Zjednodušeně, v případě použití jednoho příznaku, je vstupem modelu obraz, který
lze reprezentovat jako tenzor $X$ s tvarem $H \times W \times C$, kde $H$, $W$ a
$C$ jsou výška, šířka a kanály. V našem případě se jedná o synergické spojení
dvou sad příznaků vyplývajících z minulých sekcí, kauzálních matic $X_{COG} \in
    \mathbb{R}^{T \times T \times C}$ a časoprostorových vzorů $X_{GAF} \in
    \mathbb{R}^{T \times T \times C}$. Vstupním tenzorem je tedy příznak $X \in
    \mathbb{R}^{2 \times T \times T \times C}$. Než se vstup dostane k primární
kapsulové vrstvě, tak je provedena extrakce lokálních vlastnosti ze vstupu $X$
pomocí sady několika typů vrstev\footnote{Jednotlivé vrstvy jsou pojmenovány
    podle korespondujícího názvu v \textit{TensorFlow} a \textit{Keras} API}:
\begin{itemize}
    \item \textbf{Conv2D} --- Tato vrstva vytváří konvoluční jádro, které je
          konvolvováno se vstupem vrstvy a vytváří tenzor výstupů. V podstatě se jedná
          o sadu naučitelných filtrů. Každý filtr transformuje část obrazu
          (definovanou velikostí jádra) pomocí filtru jádra. Matice jádrového filtru
          se aplikuje na celý obraz. Filtry lze chápat jako transformaci obrazu.
    \item \textbf{BatchNormalization} --- Tato vrstva aplikuje normalizaci,
          která udržuje průměrný výstup blízko nule a směrodatnou odchylku výstupu
          blízko jedné.
    \item \textbf{MaxPool2D} --- Tato vrstva funguje jednoduše jako filtr pro
          podvzorkování. Podívá se na 2 sousední pixely a vybere maximální hodnotu.
          Slouží ke snížení výpočetní náročnosti a do jisté míry také ke snížení
          přeučení.
    \item \textbf{Dropout} --- Dropout je regularizační metoda, při níž je část
          uzlů ve vrstvě náhodně ignorována (nastaveny na nulu) pro každý
          tréninkový vzorek. Tím se náhodně vynechá část sítě a síť je nucena
          učit se funkce distribuovaným způsobem. Tato technika také zlepšuje
          generalizaci a snižuje přeučení.
\end{itemize}

\begin{figure}[h]
    \begin{center}
        \includegraphics[width=0.85\linewidth]{figures/conv}
        \caption{První část sítě ($H_{Conv}$), která mapuje vstupní obraz na
            prostor vyšší dimenze}
        \label{fig:conv}
    \end{center}
\end{figure}

Každý výstup konvoluční vrstvy $l$ se tedy skládá z konvoluční operace s určitou
rozměrovou velikostí kernelů $k$ a počtem příznakových map $f$. U konvolučních
vrstev byla použita aktivační funkce $\operatorname{ReLU}$ k přidání nelinearity
do sítě:
\begin{equation}
    F^{l+1}\left(X^l\right)=\operatorname{ReLU}\left(\text {Conv}_{k \times k}\left(X^l\right)\right)
\end{equation}
Celkově si lze první část sítě představit jako jednu funkci $H_{Conv}$, která
mapuje vstupní obraz do prostoru s vyšší dimenzí, což usnadňuje tvorbu
kapslí\footnote{\enquote{Kapsle} označuje skupinu neuronů, která společně
představuje instanci parametru specifické entity nebo části obrazu}. Tuto první
část sítě lze vidět na Obr.~\ref{fig:conv}. Následně je pak využito hloubkově
oddělitelné konvoluce, ze které je získána vrstva primárních kapslí $S_{n,d}^l$
kde $n^l$ a $d^l$ jsou počty primárních kapslí a jejich jednotlivé rozměry
$l$-té vrstvy. Základním prvkem sítě tedy již není jeden neuron, ale vektorová
výstupní kapsle, která by měla zachovávat svojí orientaci a délku. To je
realizováno pomocí \enquote{\textit{squash}} aktivační funkce:
\begin{equation}
    \operatorname{squash}\left(s_n^l\right)=\left(1-\frac{1}{e^{\left\|s_n^l\right\|}}\right) \frac{s_n^l}{\left\|s_n^l\right\|}
\end{equation}
kde $s_n^l$ označuje právě jednu kapsli. Detailně koncept kapsulární sítě popsal
Hinton~\cite{Hinton2011}. Kapsulární neuronová síť byla implementována v
programovacím jazyce Python využitím knihoven
\textit{TensorFlow}\footnote{\url{https://www.tensorflow.org}} a
\textit{Keras}\footnote{\url{https://keras.io}}. Počet filtrů konvolučních
vrstev byl zvolen 32, 64, 64 a 128 v pořadí, tak jak jdou za sebou ve
schématu~\ref{fig:conv}. Velikosti kernelů byly zvoleny 5, 3, 3 a 3. MaxPool2D
vrstvy byly přidány k zajištění kombinace lokálních rysů příznaků a učení se tak
jeho globálnějším rysům. 

\subsection{Self-attention směrování}
\label{subsec:dynamicke_smerovani}

\subsection{Marginální ztrátová funkce}
\label{subsec:marginalni_funkce}
V případě této práce se jedná o binární a vícetřídový klasifikační problém, pro
který by se za normálních okolností použila ztrátová funkce ve smyslu binární
nebo kategorické křížové entropie. Tyto ztrátové funkce ale nezachycují
sémantiku vektorů, jak je používána v kapslích. Pro tyto potřeby byla využita
marginální ztrátová funkce, kde v případě klasifikace více tříd, je pro každou
třídu reprezentovanou kapslí $n^L$ v poslední vrstvě $L$ vypočtena
pravděpodobnost existence určité třídy následovně:
\begin{equation}
    \mathcal{L}_{n^L}=T_{n^L} \max \left(0, m^{+}-\left\|u_n^L\right\|\right)^2+\lambda\left(1-T_{n^L}\right) \max \left(0,\left\|u_n^L\right\|-m^{-}\right)^2
\end{equation}
kde $T_{n^L}$ je rovno jedné, pokud je přítomna třída $n^L$, a $m^+$, $m^-$ a
$\lambda$ jsou laditelné hyperparametry. Nakonec jsou sečteny jednotlivé
ztrátové funkce $\mathcal{L}_{n^L}$ pro získání konečného \enquote{skóre} ve
fázi trénování.

\subsection{Trénování a evaluace modelů}
\label{subsec:trenovani_modelu}

\section{Statistické metody}
\label{sec:statisticke_metody}
\subsection{Lineární smíšené modely}
\label{subsec:lm_modely}

\subsection{Grangerova kauzalita}
\label{subsec:granger}
Grangerova kauzalita byla v této práci použita, k hodnocení příčinných vztahů v
rámci fyziologických signálů a k zachycení jejich interakcí v závislosti na
čase. Grangerovo pojetí vychází z myšlenky, že příčina by měla být nápomocná při
předpovídání budoucích vlivů, a to nad rámec toho, co lze předpovědět pouze na
základě jejich vlastních minulých hodnot~\cite{Granger1969}.

Formálně, časová řada $X$ je nazvána \enquote{Grangerovou kauzalitou} jiné
časové řady $Y$, jestliže regrese pro $Y$ z hlediska minulých hodnot $Y$ a $X$
je statisticky významně přesnější než regrese pouze minulých hodnot $Y$. Nechť
$\{x_t\}_{t=1}^T$ jsou zpožděné vzorky řady $X$ a $\{y_t\}_{t=1}^T$
řady $Y$ (dále jen jako vektory $\overrightarrow{x_t}$ a $\overrightarrow{y_t}$).
Poté je prvním krokem Grangerova testu následující regrese~\cite{Arnold2007}:
\begin{equation}
    \begin{gathered}
        y_t \approx A \cdot y_{t-1}+B \cdot x_{t-1}^{\vec{t}} \\
        y_t \approx A \cdot y_{t-1}
    \end{gathered}
\end{equation}
po které je možné aplikovat různé statistické testy pro získání p-hodnoty, díky
které je možné rozhodnout o výše zmíněné statistický významné přesnosti.

Běžně se Grangerova kauzalita aplikuje v rámci modelování časových řad ve smyslu
kombinatorického testování všech příznaků za účely konstrukce výstupního
příznakového kauzálního grafu (resp. kauzální matice, viz
sekce~\ref{sec:hybridni_detekce}). Takové řešení by ale pro poměrně velký počet
fyziologických příznaků bylo extrémně výpočetně náročné, a proto bylo dále
uznáno za nevhodné.

Řešením se zde naskytla regrese, kterou lze využít pro identifikaci podmnožiny
příznaků, na které je daný příznak podmíněně závislý. To vychází z faktu, že
nejlepší regresor pro danou proměnnou s nejmenší kvadratickou chybou bude mít
teoreticky nenulové koeficienty pouze pro proměnné v okolí\footnote{Statisticky
    \enquote{okolí} proměnné implikuje podmnožinu proměnných, které s ní úzce
    souvisejí.}~\cite{Schindler2013,Arnold2007}. Pro tento regresní problém byl
zvolen Lasso algoritmus, jež je uveden do souvislosti v následující sekci.

\subsection{Lasso regrese}
\label{subsec:lasso}
\gls{Lasso} (Least Absolute Shrinkage and Selection Operator) je široce
používaná technika lineární regrese pro výběr a regularizaci proměnných využitím
$\ell_1$ penalizačního členu. Formálně, výstup $\vec{w}$ minimalizuje součet
průměrné kvadratické chyby regrese pro $y$:
\begin{equation}
    \vec{w}=\arg \min \frac{1}{n} \sum_{(\vec{x}, y) \in X}|\vec{w} \cdot \vec{x}-y|^2+\lambda\|\vec{w}\|_1
\end{equation}
kde $X$ je vstupní příznak, $n$ je počet vzorků v $X$ a $\lambda$ je penalizační
člen určující míru regularizace koeficientů. S rostoucí hodnotou $\lambda$ se
více koeficientů smršťuje směrem k nule, což vede k řídkému modelu s menším
počtem prediktorů a naopak~\cite{Tibshirani1996}. Ve smyslu tvorby kauzálních
matic poskytuje Lasso množinu časových proměnných, které při regresi $y_t$ podle
zpožděných proměnných $x_{t'}$, kde $t'=\{t-T,...,t -1\}$ pro všechna $x \in X$,
nabývají právě nenulového koeficientu Grangerovy kauzality.

Nicméně Bahadori a Liu dokázali v~\cite{Bahadori2013}, že Grangerova kauzalita
je v rámci použití vícerozměrných dat inkonzistentní a není dobře schopna
zachytit nelineární vztahy nebo složité struktury závislostí. Vzhledem k tomu,
že data využívané v této práci jsou vícerozměrná a reprezentují biosignály, v
rámci kterých mohou existovat právě nelineární vztahy nebo komplexní
hierarchické relace, byl využit přístup Kopula-Granger. S využitím Lasso ukázali
v~\cite{Bahadori2013} jeho konzistenci na vícerozměrných datech i jeho schopnost
efektivně zachytit nelinearitu v datech (viz
sekce~\ref{subsec:kauzalni_matice}).

\subsection{Teorie kopulí}
\label{subsec:teorie_kopul}
Vzhledem k využití Kopula-Granger přístupu pro tvorbu vícerozměrných kauzálních
matic je žádoucí stručně představit kopula funkce. V teorii pravděpodobnosti a
statistice je kopula pravděpodobnostní distribuční funkcí, jež popisuje
závislost mezi jednotlivými marginálními distribucemi a poskytuje způsob jak
modelovat právě společnou distribuci náhodných veličin bez specifikace samotných
distribucí těchto veličin. Základem teorie kopulí je Sklarův teorém, který říká,
že jakákoli více-dimenzionální distribuce může být zapsána jako kopula
aplikovaná na její marginální distribuce:

\begin{theorem}[Sklarův teorém]
    \label{theorem:sklar}
    Nechť $F$ je vícerozměrná distribuční funkce s marginálními distribucemi
    $F_1, F_2, \ldots, F_n$. Pak existuje kopula $C$ taková, že:
    \begin{equation}
        F\left(x_1, x_2, \ldots, x_n\right)=C\left(F_1\left(x_1\right), F_2\left(x_2\right), \ldots, F_n\left(x_n\right)\right)
    \end{equation}
    Kopula $C$ je jednoznačná, pokud marginální distribuce $F_1, F_2, \ldots, F_n$ jsou spojité.
\end{theorem}

Kopula-Granger model, ve kterém jsou v této práci kopula funkcí mapovány
marginální distribuce fyziologických událostí do kopula prostoru je popsán v
sekci~\ref{subsec:kauzalni_matice}. Pro shrnutí vychází tento model z
následujících kroků~\cite{Guy2016}:
\begin{enumerate}
    \item Nalezení empirického marginálního rozdělení pro fyziologickou událost
          $\hat{F_i}$.
    \item Mapování pozorování do kopula prostoru:
          $\hat{f}_i\left(x_t^i\right)=\hat{\mu}_i+\hat{\sigma}_i.
          \Phi^{-1}\left(\hat{F}_i\left(x_t^i\right)\right)$.
    \item Nalezení Grangerovy kauzality v rámci $\hat{f}_i\left(x_t^i\right)$.
\end{enumerate}
přičemž je dále brán v potaz Winsorizovaný\footnote{Winsorizace nebo Winsorova
transformace je transformace statistických dat omezením extrémních hodnot, aby
se snížil vliv případných odlehlých hodnot} odhad použité distribuční funkce,
podle~\cite{Bahadori2013}, aby se zabránilo velkým číslům
$\Phi^{-1}\left(0^{+}\right)$\footnote{$\Phi^{-1}$ je inverzní kumulativní
distribuční funkce standardního normálního rozdělení.} and
$\Phi^{-1}\left(1^{-}\right)$:
\begin{equation}
    \tilde{F}_j= \begin{cases}\delta_n, & \text { if } \hat{F}\left(x^j\right)<\delta_n \\ \hat{F}\left(x^j\right) & \text { if } \delta_n \leq \hat{F}\left(x^j\right)<1-\delta_n \\ \left(1-\delta_n\right) & \text { if } \hat{F}\left(x^j\right)>1-\delta_n .\end{cases}
\end{equation}

\subsection{Metriky hodnocení v strojovém učení}
\label{subsec:ml_metriky}




% \section{}
% \label{sec:}
% \input{}

% \section{Použité technologie a knihovny}
% \label{sec:technologie_a_knihovny}
% Obory umělé inteligence, jako strojové učení nebo neuronové sítě, často vyžadují
v reálných podmínkách pečlivou přípravu a předzpracování dat nebo sestavení a
trénování modelů. V dnešní době však existuje velké množství nástrojů a
knihoven, které tyto kroky implementují a značně tak zvyšují efektivitu vývoje
patřičných aplikací. Tato kapitola popisuje zásadní nástroje použité pro účely
této práce.

\subsection{Python a R}
\label{subsec:python_r}
Mezi nejpopulárnější open-source programovací jazyky v oblasti strojového učení
a data science, které byly zároveň použity v této práci, patří
Python\footnote{https://www.python.org} a R\footnote{https://www.r-project.org}.
Pro předzpracování dat, strojové učení a neuronové sítě byl použit
Python 3.7 s využitím platformy Google Colab.

Explorační a statistická analýza dat byla realizována prostřednictvím jazyka R
(verze 4.2.1, Funny-Looking Kid) na laptopu \textit{HP Spectre x360} s
procesorem \textit{i7-8705G}, 32~GB DDR4 RAM a grafickou kartou \textit{RX Vega
M GL}. I přestože R není na rozdíl od Pythonu univerzálním vysokoúrovňovým
programovacím jazykem a využívá se především pro statistické modelování, je díky
bohaté komunitě a velkému množství knihoven nedílnou součástí oblasti strojového
učení a data science. 

\subsection{Google Colab a Jupyter Notebook}
\label{subsec:jupyter_colab}
Na základě velkého objemu dat ke zpracování bylo využito platformy Google
Colab\footnote{https://colab.research.google.com}. Jedná se o cloudové
interaktivní výpočetní prostředí, které běží na virtuálním stroji a umožňuje
vzdálené spuštění kódu s využitím prostředků jako \textit{NVIDIA Tesla
V100/P100} s 24~GB VRAM. Jinými slovy jde o hostovanou webovou aplikaci jménem
Jupyter Notebook\footnote{https://jupyter.org}, která umožňuje vytvářet a sdílet
dokumenty (zápisníky). Tyto dokumenty jsou rozděleny do buněk, které lze
spouštět v libovolném pořadí (live kód), což zajišťuje efektivnější
prototypování.

\subsection{Neurokit}
\label{subsec:neurokit}
Knihovna Neurokit2~\cite{Makowski2021neurokit} poskytuje pokročilé metody pro
zpracování a vizualizaci biosignálu. Jednotlivé metody zároveň nabízejí možnost
si vybrat z mnoha implementovaných algoritmů, například pro detekci QRS
komplexu. V této práci byla knihovna použita pro předzpracování respirační,
elektrodermální a srdeční aktivity včetně zpracování HRV.

\subsection{Tidyverse a Easystats}
\label{subsec:tidyverse_easystats}
Knihovny tidyverse~\cite{tidyverse} a easystats~\cite{easystats} rozšiřují jazyk
R o mnoho funkcionalit primárně pro potřeby statistického modelování a
strojového učení. Usnadňují a zrychlují proces tvorby modelů díky dobře
zdokumentovanému ekosystému balíčků. V této práci sloužili knihovny ke
statistické analýze velkého souboru dat. 

\subsection{Scikit-learn, TensorFlow a Keras}
\label{subsec:scitkit_tensor_keras}
Scikit-learn~\cite{sklearn_api} je balíček jazyka Python pro prediktivní analýzu
dat a strojové učení, který byl v této práci použit pro extrakci a normalizaci
příznaků. Dále pro porovnávání, validaci a výběr parametrů a modelů.

% TensorFlow is an open-source framework developed at Google for machine learning
% applications. Its main focus is on defining the architecture and training of
% deep neural networks. It is highly optimized for the execution of low level
% tensor operations on CPU, GPU, or TPU. 

% Keras is a high-level API that acts as an interface for the TensorFlow
% framework. It enables faster prototyping of ANNs by providing abstractions and
% building blocks for developing the models. It also provides the implementation
% of several popular CNN architectures along with their weights, making transfer
% learning more accessible. The simplest way of defining Keras models is by using
% the Sequential model API, which is essentially a linear stack of defined layers.
% The alternative is adopting the Keras functional API, which allows for building
% arbitrary graphs of layers with multiple inputs and outputs or using residual
% skipping connections.

\subsection{InfluxDB}
\label{subsec:influx}
InfluxDB\footnote{https://www.influxdata.com} je open-source platforma
poskytující databázi pro časové řady. Zahrnuje rozhraní (API) pro standardní
databázové dotazy. Součástí je i grafické uživatelské rozhraní (GUI) s
modulárními uživatelskými panely pro monitorování dat v reálném čase. Tato
platforma (InfluxDB OSS 2.4) byla využita v rámci experimentální části práce k
uchovávání a vizualizaci dat.



% \section{Statistické metody}
% \label{sec:_statisticke_metody}
% \subsection{Lineární smíšené modely}
\label{subsec:lm_modely}

\subsection{Grangerova kauzalita}
\label{subsec:granger}
Grangerova kauzalita byla v této práci použita, k hodnocení příčinných vztahů v
rámci fyziologických signálů a k zachycení jejich interakcí v závislosti na
čase. Grangerovo pojetí vychází z myšlenky, že příčina by měla být nápomocná při
předpovídání budoucích vlivů, a to nad rámec toho, co lze předpovědět pouze na
základě jejich vlastních minulých hodnot~\cite{Granger1969}.

Formálně, časová řada $X$ je nazvána \enquote{Grangerovou kauzalitou} jiné
časové řady $Y$, jestliže regrese pro $Y$ z hlediska minulých hodnot $Y$ a $X$
je statisticky významně přesnější než regrese pouze minulých hodnot $Y$. Nechť
$\{x_t\}_{t=1}^T$ jsou zpožděné vzorky řady $X$ a $\{y_t\}_{t=1}^T$
řady $Y$ (dále jen jako vektory $\overrightarrow{x_t}$ a $\overrightarrow{y_t}$).
Poté je prvním krokem Grangerova testu následující regrese~\cite{Arnold2007}:
\begin{equation}
    \begin{gathered}
        y_t \approx A \cdot y_{t-1}+B \cdot x_{t-1}^{\vec{t}} \\
        y_t \approx A \cdot y_{t-1}
    \end{gathered}
\end{equation}
po které je možné aplikovat různé statistické testy pro získání p-hodnoty, díky
které je možné rozhodnout o výše zmíněné statistický významné přesnosti.

Běžně se Grangerova kauzalita aplikuje v rámci modelování časových řad ve smyslu
kombinatorického testování všech příznaků za účely konstrukce výstupního
příznakového kauzálního grafu (resp. kauzální matice, viz
sekce~\ref{sec:hybridni_detekce}). Takové řešení by ale pro poměrně velký počet
fyziologických příznaků bylo extrémně výpočetně náročné, a proto bylo dále
uznáno za nevhodné.

Řešením se zde naskytla regrese, kterou lze využít pro identifikaci podmnožiny
příznaků, na které je daný příznak podmíněně závislý. To vychází z faktu, že
nejlepší regresor pro danou proměnnou s nejmenší kvadratickou chybou bude mít
teoreticky nenulové koeficienty pouze pro proměnné v okolí\footnote{Statisticky
    \enquote{okolí} proměnné implikuje podmnožinu proměnných, které s ní úzce
    souvisejí.}~\cite{Schindler2013,Arnold2007}. Pro tento regresní problém byl
zvolen Lasso algoritmus, jež je uveden do souvislosti v následující sekci.

\subsection{Lasso regrese}
\label{subsec:lasso}
\gls{Lasso} (Least Absolute Shrinkage and Selection Operator) je široce
používaná technika lineární regrese pro výběr a regularizaci proměnných využitím
$\ell_1$ penalizačního členu. Formálně, výstup $\vec{w}$ minimalizuje součet
průměrné kvadratické chyby regrese pro $y$:
\begin{equation}
    \vec{w}=\arg \min \frac{1}{n} \sum_{(\vec{x}, y) \in X}|\vec{w} \cdot \vec{x}-y|^2+\lambda\|\vec{w}\|_1
\end{equation}
kde $X$ je vstupní příznak, $n$ je počet vzorků v $X$ a $\lambda$ je penalizační
člen určující míru regularizace koeficientů. S rostoucí hodnotou $\lambda$ se
více koeficientů smršťuje směrem k nule, což vede k řídkému modelu s menším
počtem prediktorů a naopak~\cite{Tibshirani1996}. Ve smyslu tvorby kauzálních
matic poskytuje Lasso množinu časových proměnných, které při regresi $y_t$ podle
zpožděných proměnných $x_{t'}$, kde $t'=\{t-T,...,t -1\}$ pro všechna $x \in X$,
nabývají právě nenulového koeficientu Grangerovy kauzality.

Nicméně Bahadori a Liu dokázali v~\cite{Bahadori2013}, že Grangerova kauzalita
je v rámci použití vícerozměrných dat inkonzistentní a není dobře schopna
zachytit nelineární vztahy nebo složité struktury závislostí. Vzhledem k tomu,
že data využívané v této práci jsou vícerozměrná a reprezentují biosignály, v
rámci kterých mohou existovat právě nelineární vztahy nebo komplexní
hierarchické relace, byl využit přístup Kopula-Granger. S využitím Lasso ukázali
v~\cite{Bahadori2013} jeho konzistenci na vícerozměrných datech i jeho schopnost
efektivně zachytit nelinearitu v datech (viz
sekce~\ref{subsec:kauzalni_matice}).

\subsection{Teorie kopulí}
\label{subsec:teorie_kopul}
Vzhledem k využití Kopula-Granger přístupu pro tvorbu vícerozměrných kauzálních
matic je žádoucí stručně představit kopula funkce. V teorii pravděpodobnosti a
statistice je kopula pravděpodobnostní distribuční funkcí, jež popisuje
závislost mezi jednotlivými marginálními distribucemi a poskytuje způsob jak
modelovat právě společnou distribuci náhodných veličin bez specifikace samotných
distribucí těchto veličin. Základem teorie kopulí je Sklarův teorém, který říká,
že jakákoli více-dimenzionální distribuce může být zapsána jako kopula
aplikovaná na její marginální distribuce:

\begin{theorem}[Sklarův teorém]
    \label{theorem:sklar}
    Nechť $F$ je vícerozměrná distribuční funkce s marginálními distribucemi
    $F_1, F_2, \ldots, F_n$. Pak existuje kopula $C$ taková, že:
    \begin{equation}
        F\left(x_1, x_2, \ldots, x_n\right)=C\left(F_1\left(x_1\right), F_2\left(x_2\right), \ldots, F_n\left(x_n\right)\right)
    \end{equation}
    Kopula $C$ je jednoznačná, pokud marginální distribuce $F_1, F_2, \ldots, F_n$ jsou spojité.
\end{theorem}

Kopula-Granger model, ve kterém jsou v této práci kopula funkcí mapovány
marginální distribuce fyziologických událostí do kopula prostoru je popsán v
sekci~\ref{subsec:kauzalni_matice}. Pro shrnutí vychází tento model z
následujících kroků~\cite{Guy2016}:
\begin{enumerate}
    \item Nalezení empirického marginálního rozdělení pro fyziologickou událost
          $\hat{F_i}$.
    \item Mapování pozorování do kopula prostoru:
          $\hat{f}_i\left(x_t^i\right)=\hat{\mu}_i+\hat{\sigma}_i.
          \Phi^{-1}\left(\hat{F}_i\left(x_t^i\right)\right)$.
    \item Nalezení Grangerovy kauzality v rámci $\hat{f}_i\left(x_t^i\right)$.
\end{enumerate}
přičemž je dále brán v potaz Winsorizovaný\footnote{Winsorizace nebo Winsorova
transformace je transformace statistických dat omezením extrémních hodnot, aby
se snížil vliv případných odlehlých hodnot} odhad použité distribuční funkce,
podle~\cite{Bahadori2013}, aby se zabránilo velkým číslům
$\Phi^{-1}\left(0^{+}\right)$\footnote{$\Phi^{-1}$ je inverzní kumulativní
distribuční funkce standardního normálního rozdělení.} and
$\Phi^{-1}\left(1^{-}\right)$:
\begin{equation}
    \tilde{F}_j= \begin{cases}\delta_n, & \text { if } \hat{F}\left(x^j\right)<\delta_n \\ \hat{F}\left(x^j\right) & \text { if } \delta_n \leq \hat{F}\left(x^j\right)<1-\delta_n \\ \left(1-\delta_n\right) & \text { if } \hat{F}\left(x^j\right)>1-\delta_n .\end{cases}
\end{equation}

\subsection{Metriky hodnocení v strojovém učení}
\label{subsec:ml_metriky}




% \begin{figure}[ht]
%     \centering
%     \begin{subfigure}[b]{0.45\textwidth}
%       \mybox{dirs}{%
%       \dirtree{%
%       .1 COVIDx8B.
%       .2 labels.
%       .3 train\_COVIDx8B.txt.
%       .3 test\_COVIDx8B.txt.
%       .2 train.
%       .3 Image\_1.
%       .3 Image\_2.
%       .3 \vdots.
%       .2 test.
%       .3 Image\_1.
%       .3 Image\_2.
%       .3 \vdots.
%       }
%       \tcblower
%       \caption{Original COVIDx8B}
%       \label{subfig:tree1}
%       }
%     \end{subfigure}
%     \begin{subfigure}[b]{0.45\textwidth}
%       \mybox{dirs}{%
%       \dirtree{%
%       .1 COVIDx8B.
%       .2 train.
%       .3 negative.
%       .4 Image\_1.
%       .4 Image\_2.
%       .4 \vdots.
%       .3 positive.
%       .4 Image\_1.
%       .4 Image\_2.
%       .4 \vdots.
%       .2 test.
%       .3 negative.
%       .4 Image\_1.
%       .4 Image\_2.
%       .4 \vdots.
%       .3 positive.
%       .4 Image\_1.
%       .4 Image\_2.
%       .4 \vdots.
%       }
%       \tcblower
%       \caption{Preprocessed COVIDx8B}
%       \label{subfig:tree2}
%       }
%     \end{subfigure}
%     \caption[Comparison of the directory tree of the original COVIDx8B dataset and our preprocessed version]{Comparison of the directory tree of the original COVIDx8B dataset (a) as used by the COVID-Net project, and the preprocessed version (b) which was used in our experiments for image generation during training.}
%     \label{fig:directory_comparison}
%   \end{figure}
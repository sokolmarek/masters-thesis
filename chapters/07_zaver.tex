V diplomové práci byla realizována evaluace vlivu extrémního prostředí na
projevy kognitivní zátěže v biosignálech. Byla analyzována data získaná v
průběhu analogové vesmírné mise, studující vliv izolace na člověka. Pro účely
analýzy byly využity popisné statistiky, vizuální metody v podobě bodových
rozsahů nebo krabicových grafů a statistické modelování. Výsledky byly srovnány
v rámci prostředí s vysokým a standardním atmosférickým tlakem. 

Dále byl hodnocen vliv úloh navozujících zvýšenou kognitivní zátěž na
elektrickou srdeční, dechovou a elektrodermální aktivitu společně s využitím
časových, frekvenčních a nelineárních parametrů sledovaných signálů. Pro
hodnocení bylo využito popisných statistik, krabicových grafů a navržené metody
využívající metody strojového učení. Výsledky byly srovnány jak na úrovni
posádek mateřské lodi a přistávacího modelu, tak i mezi jednotlivými subjekty.

Pro účely samotného hodnocení kognitivní zátěže bylo navrženo a realizováno
multimodální řešení ve formě kapsulární neuronové sítě. Byla navržena metodika
pro tvorbu fyziologických příznaků jakožto vstup pro neuronovou síť. Realizované
řešení bylo otestováno na veřejně dostupných benchmarkovacích datasetech a
srovnáno se současným stavem. 

\section{Budoucí práce}
V rámci projektu Hydronaut jsou chystány další vesmírné analogové mise, během
kterých bude probíhat sběr dat. Tato data budou následně využita pro budoucí
experimenty a širší statistické analýzy. Zároveň budou sloužit pro vývoj nových
metod, například evaluace mentálního stavu posádky, které můžou být v
budoucnosti kritické pro úspěšné průběhy dlouhodobých vesmírných misí. Dále bude
rozšiřována navržená metodika hodnocení kognitivní zátěže o experimenty s
metodami předzpracování a standardizace, optimalizací hyperparametrů nebo
augmentací dat. 

Ohled bude brán také na problematiku předzpracování biosignálů, ať už z hlediska
návrhu systematického porovnání a výběru QRS detektorů nebo realizace nových
řešení. Jedním z takových řešení by mohlo být převzetí koncepce ze statistiky a
strojového učení, konkrétně ansámblového učení (ensemble learning). Metody by se
tak mohly zkombinovat. Nejdříve by se ale musely některé vyřadit, jelikož
algoritmy jako~\cite{Tompkins1985,Christov2004} (založené na
integraci/prahování) nejsou ve skutečnosti detektory R vln. Jedná se o techniky
založené na detekci QRS komplexu a vedou k horšímu temporálnímu rozlišení polohy
R vln~\cite{Porr2019}. Stěžejní v tomto přístupu je také nejednoznačnost
kombinace výsledků metod do jednoho určitého pravděpodobnostního výroku.
Výsledkem metod jsou totiž ve většině případech indexy detekovaných R vln v
signálu. 

Dalším směrem budoucího výzkumu bude dále návrh postupů statistického hodnocení
experimentů v extrémním prostředí s ohledem na jejich zásadní omezení. Cestou by
mohlo být oproštění se od frekventistického přístupu zaměřeného na testování
hypotéz, u něhož se také vyskytuje problém s nesprávným používáním p-hodnot,
který kriticky přispívá ke krizi
reprodukovatelnosti~\cite{Chambers2014,Szucs2016}. Postupem let čím dál více
panuje přesvědčení, že zobecnění Bayesovského přístupu je jedním ze způsobů, jak
tyto problémy překonat~\cite{Benjamin2018}. Bayesovský rámec pro statistiku
rychle získává na popularitě, a to nejen díky technologickým
pokrokům~\cite{Winkler2001,Ferreira2020}. Mezi důvody proč toto paradigma
upřednostnit patří například: spolehlivost, přesnost (i v případě zašuměných dat
a malých vzorků), možnost zahrnutí předchozích znalostí a intuitivní charakter
výsledků s jejich jednoduchou
interpretací~\cite{Etz2016,Kruschke2012,Kruschke2010,Winkler2001}. 
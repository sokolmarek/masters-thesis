Sledování a hodnocení kognitivní zátěže v extrémním prostředí, jako je například
vesmír, je nesmírně důležité pro zajištění bezpečnosti a úspěchu jednotlivců a
týmů působících v takovém prostředí během náročných a důležitých úkonů.
Monitorování kognitivních funkcí pomocí tradičních metod, jako je dotazníkové
hodnocení nebo behaviorální analýza, může být v izolovaném, uzavřeném a
extrémním prostředí nepraktické až neproveditelné kvůli například komunikačním a
časovým omezením. Proto čím dál více vzrůstá zájem o využití periferních
biosignálů, jako je například elektrická srdeční aktivita nebo kožní vodivost, k
hodnocení kognitivní zátěže v reálném čase. Interpretace těchto biosignálů v
souvislosti s kognitivní zátěží je však nejasná a náročná vzhledem k tomu, že se
do periferních biosignálu promítají veškeré kognitivní funkce. 

Klíčem k pochopení by mohla být inkorporace komplexního modelu neuroviscerální
integrace, který předpokládá, že autonomní nervový systém hraje klíčovou roli v
regulaci kognitivních procesů a poskytuje teoretický rámec pro pochopení vztahu
mezi periferními biosignály s kognitivní zátěží. Obecně tento model předpokládá,
že kognitivní zátěž několikaúrovňově moduluje aktivitu autonomního nervového
systému, což vede ke změnám periferní fyziologie.

Navzdory potenciálu periferních biosignálů jako indikátorů kognitivní zátěže v
extrémních prostředích nadále zůstává při interpretaci jejich vlastností několik
problémů. Patří mezi ně individuální rozdíly v charakteru biosignálů, vliv
faktorů prostředí a nedostatečná standardizace sběru a analýzy dat. Řešení
těchto problémů je zásadní pro vývoj spolehlivých a platných metod kognitivní
zátěže.

Tato práce se zabývá využitím periferních biosignálů pro hodnocení kognitivní
zátěže v extrémním prostředí. Konkrétně analyzovanými periferními biosignály
jsou srdeční, elektrodermální a respirační aktivita. Dále se také práce zabývá
problémy spojenými s interpretací příznaků vybraných biosignálů a navrhuje možná
řešení k jejich překonání v souvislosti s modelem neuroviscerální integrace.


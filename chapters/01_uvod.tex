Sledování a hodnocení kognitivní zátěže v extrémním prostředí, jako je například
vesmír, je kritické pro zajištění bezpečnosti a úspěchu jednotlivců a týmů
působících v takovém prostředí během náročných a důležitých úkonů. Monitorování
kognitivních funkcí pomocí tradičních metod, jako je dotazníkové hodnocení nebo
behaviorální analýza, může být v izolovaném, uzavřeném a extrémním prostředí
nepraktické až neproveditelné kvůli například komunikačním a časovým omezením.
Proto čím dál více vzrůstá zájem o využití periferních biosignálů, jako je
například elektrická srdeční nebo elektrodermální aktivita, k hodnocení
kognitivní zátěže v reálném čase. 

Paradigma většiny dosavadních přístupů detekce těží primárně z extrahovaných
příznaků v podobě statistických ukazatelů nebo často například parametrů
vypočtených z variability srdeční frekvence. Interpretace těchto parametrů v
souvislosti s kognitivní zátěží je však nejasná a náročná vzhledem k tomu, že se
do periferních biosignálu promítají veškeré kognitivní funkce. Velké množství
těchto parametrů také vyžaduje poměrně dlouhé časové úseky pro jejich výpočet a
jsou velmi citlivé na různorodé typy artefaktů.

Jako klíčem k interpretaci těchto vztahů se jeví inkorporace komplexního modelu
neuroviscerální integrace, který vychází ze skutečnosti, že autonomní nervový
systém hraje klíčovou roli v regulaci periferní fyziologie a poskytuje
teoretický rámec pro pochopení vztahů mezi periferními biosignály s kognitivními
funkcemi. Obecně tento model předpokládá, že kognitivní zátěž několikaúrovňově
moduluje aktivitu autonomního nervového systému, což vede právě ke změnám
periferní fyziologie. Rozklíčování interpretace takovým způsobem by ale
vyžadovalo velké množství patřičně popsaných dat, bylo velmi časově náročné a
pravděpodobně by trpělo nedostatečnou generalizací.

Nedostatky předešlého přístupu překonávají metody hlubokého učení, jejichž
vstupem jsou samotné biosignály, ze kterých jsou potřebné příznaky pro účely
detekce kognitivní zátěže extrahovány automaticky. Takové typy detektorů jsou
často robustní a schopny v datech nalézt nespočetné množství relací. Ve výsledku
je tedy pro účely rozpoznání kognitivní zátěže extrahováno a použito ohromné
množství parametrů. Nevýhodou tohoto přístupu je tím pádem často vysoká
výpočetní náročnost, a zároveň jeho prisma, které nijak nezohledňuje kauzalitu
mezi použitými fyziologickými příznaky, naopak vychází z předpokladu jejich
nezávislosti. Tento fakt znovu činí interpretaci a determinismus zásadních
okolností v rámci hodnocení kognitivní zátěže stěžejními. Obecně se mezi další
problémy také řadí například individuální rozdíly v charakteru biosignálů, vliv
faktorů prostředí nebo nedostatečná standardizace sběru a analýzy dat. Řešení
těchto problémů je také zásadní pro vývoj spolehlivých a platných metod pro
hodnocení kognitivní zátěže.

Tato práce představuje nový způsob hodnocení kognitivní zátěže pomocí takzvaných
vícerozměrných časoprostorových kauzálních vzorů společně s využitím kapsulární
sítě. Tyto vzory zachycují temporální příčinné relace a umožňují unikátní
kódování specifického kognitivního stavu v podobě charakteristické konfigurace v
prostorové doméně.




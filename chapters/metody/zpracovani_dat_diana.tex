\subsection{Zpracování exportovaných segmentů biosignálů}
\label{subsec:prezpracovani_segmentu}
Pro každého člena posádky byly exportovány segmenty biosignálů o délce 30s s
50\% překryvem na základě poznatků
v~\cite{Castaldo2019,Kim2021,Pecchia2018,Shaffer2020,Tervonen2021}. Zpracování
biosignálu vycházelo z metodiky popsané v sekci~\ref{sec:zpracovani_biosignalu}.
U každého segmentu proběhlo hodnocení jeho kvality podle dvou kritérií:
\begin{itemize}
    \item Hodnocení kvality \gls{EKG} signálu pomocí heuristické fúze a fuzzy
    komplexního hodnocení podle~\cite{Zhao2018}.
    \item Hodnocení detekovaných R vln z hlediska časové kontroly náhlých
    nefyziologických změn v po sobě jdoucích R-R intervalech.
\end{itemize}
Segmenty které vykazovali nežádoucí anomálie v rámci hodnotících kritérií byly
vyřazeny. Ze segmentů bylo dále vypočteno následně přes 100 různých parametrů
pro účely analýzy dat. Mezi tyto parametry patřily například běžné statistické
charakteristiky (průměr, medián, směrodatná odchylka a další) nebo nelineární a
časové \gls{HRV} parametry. Zpracované segmenty byly zároveň anotovány, a to z
hlediska spánkového cyklu. Dále byly identifikovány a označeny v časech
kognitivních testů. Z časových důvodu nebyly pro účely této práce ostatní
aktivity během mise anotovány, i přes dostupnost kamerových záznamů. Seznam
všech počítaných parametrů je součástí přílohy v souboru
\texttt{all\_params.csv}.

\subsection{Čistění dat}
\label{subsec:cisteni_dat}
Ze souborů vypočtených parametrů byly vynechány všechny parametry, jejichž
sloupce obsahovali \texttt{NaN} hodnoty. Dále byly parametry korelovány a
odstranili se ty, které byly vzájemně dokonale korelované ($|r| > 0,999$). Poté
se odstranili odlehlé hodnoty na základě absolutní odchylky mediánu od mediánu.

\subsection{Sledované veličiny}
\label{subsec:sledovane_veliciny}
Pro účely analýzy \gls{NPF} adaptace probandů v průběhu mise byly vybrány a
sledovány především následující \gls{HRV} parametry:

\subsection{Tvorba hypotetických LMM modelů}
\label{subsec:tvorba_modelů}


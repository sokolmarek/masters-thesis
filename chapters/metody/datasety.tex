\subsection{WESAD}
\label{subsec:wesad}
Dataset WESAD~\cite{wesadDataset} (\textit{Wearable Stress and Affect
Detection}) je multimodální dataset navržený pro výzkum v oblasti hodnocení
stresu a emocí za použití nositelných senzorů. Tento dataset byl vytvořen s
cílem přispět k vývoji pokročilých algoritmů strojového učení pro analýzu
fyziologických signálů a rozpoznání emocí. WESAD obsahuje data získaná od 15
účastníků, přičemž každý z nich prošel sérií experimentů v laboratorních
podmínkách.

Data byla v datasetu získána ze dvou nositelných zařízení. Prvním zařízením byl
\textit{RespiBAN}\footnote{Zařízení \textit{RespiBAN} již není vyráběno},
nositelný senzor umístěný na hrudi se vzorkovací frekvencí 700~Hz, který
zaznamenával elektrokardiogram, elektrodermální aktivitu, elektromyogram,
respirační signál a teplotu těla. Druhé zařízení byl chytrý náramek
\textit{Empatica E4}\footnote{\url{https://www.empatica.com/research/e4}}, který
zaznamenává krevní tlak (64~Hz), elektrodermální aktivitu (4~Hz), teplotu těla
(4~Hz) a tříosou akceleraci (32~Hz). Experiment, který byl proveden pro sběr
dat, zahrnoval celkem čtyři fáze:
\begin{enumerate}
    \item \textbf{Základní úroveň (Baseline condition)} --- účastník byl požádán, aby
    seděl/stál v klidu po dobu 20 minut.
    \item  \textbf{Stresový úkol (Stress condition)} --- účastník musel pět minut
    přednášet před publikem a poté vyřešit matematický úkol, který byl navržen
    tak, aby vyvolal stres.
    \item  \textbf{Relaxační úkol (Amusement condition)} --- účastník sledoval
    komediální video, které mělo vyvolat příjemné emoce.
    \item  \textbf{Řízená meditace (Meditation)} --- účastník prováděl řízenou
    meditaci, jejíž cílem bylo navození do stavu blízkého neutrálnímu
    afektivnímu stavu.
\end{enumerate}

WESAD poskytuje časově synchronizovaná, předzpracovaná a anotovaná data z těchto
nositelných senzorů. Pro účely diplomové práce byly použity signály ze zařízení
\textit{RespiBAN}, konkrétně srdeční, respirační a elektrodermální aktivita. Pro
úlohy hodnocení kognitivní zátěže byla základní úroveň označena jako klidový stav
a stresové úkoly byly označeny jako stav kognitivní zátěže.

\subsection{CLAS}
\label{subsec:clas}
Dataset CLAS~\cite{clasDataset} (\textit{Cognitive Load, Affect, and Stress
Recognition}) je podobně jako WESAD multimodální dataset vytvořený pro výzkum v
oblasti rozpoznávání kognitivní zátěže a emocí za použití nositelných senzorů a
dalších datových zdrojů. Dataset zahrnuje data získaná od 62 účastníků, kteří
byli vystaveni různým úkolům a podnětům navrženým tak, aby vyvolaly různé úrovně
kognitivní zátěže a emocí. Mezi tyto podněty patřily například matematické úlohy
nebo Stroopův test. Každý účastník byl zároveň měřen i v klidu (dataset stav
označuje jako Baseline). Během přechodů mezi jednotlivými úkoly bylo účastníkům
puštěno neutrální video nebo byl účastník požádán, aby vyplnil dotazník (dataset
označuje jako Neutral). 

Fyziologická data v rámci tohoto datasetu byla měřena zařízením
\textit{Shimmer3}\footnote{\url{https://shimmersensing.com}} se vzorkovací
frekvencí 256~Hz. Mezi měřené biologické signály patří elektrokardiogram,
elektrodermální aktivita a fotopletysmogram. Pro úlohy hodnocení kognitivní
zátěže byl Baseline stav označen jako klidový stav a stresové úkoly byly
označeny jako stav kognitivní zátěže. 

\subsection{Data z vesmírné analogové mise DIANA}
\label{subsec:data_diana}
Pro potřeby diplomové práce poskytla \gls{FF UPOL} data z vesmírné analogové
mise DIANA. Součástí dat jsou osmidenní 24 hodinové záznamy biologických
signálů, kamerových záznamů a anotace v podobě časů kognitivních úloh pro
každého člena posádek. Využitými signály v této práci jsou elektrokardiogram
spolu s elektrodermální a respirační aktivitou.
\subsection{Lineární smíšené modely}
\label{subsec:lm_modely}

\subsection{Grangerova kauzalita}
\label{subsec:granger}
Grangerova kauzalita byla v této práci použita, k hodnocení příčinných vztahů v
rámci fyziologických signálů a k zachycení jejich interakcí v závislosti na
čase. Grangerovo pojetí vychází z myšlenky, že příčina by měla být nápomocná při
předpovídání budoucích vlivů, a to nad rámec toho, co lze předpovědět pouze na
základě jejich vlastních minulých hodnot~\cite{Granger1969}.

Formálně, časová řada $X$ je nazvána \enquote{Grangerovou kauzalitou} jiné
časové řady $Y$, jestliže regrese pro $Y$ z hlediska minulých hodnot $Y$ a $X$
je statisticky významně přesnější než regrese pouze minulých hodnot $Y$. Nechť
$\{x_t\}_{t=1}^T$ jsou zpožděné vzorky řady $X$ a $\{y_t\}_{t=1}^T$
řady $Y$ (dále jen jako vektory $\overrightarrow{x_t}$ a $\overrightarrow{y_t}$).
Poté je prvním krokem Grangerova testu následující regrese~\cite{Arnold2007}:
\begin{equation}
    \begin{gathered}
        y_t \approx A \cdot y_{t-1}+B \cdot x_{t-1}^{\vec{t}} \\
        y_t \approx A \cdot y_{t-1}
    \end{gathered}
\end{equation}
po které je možné aplikovat různé statistické testy pro získání p-hodnoty, díky
které je možné rozhodnout o výše zmíněné statistický významné přesnosti.

Běžně se Grangerova kauzalita aplikuje v rámci modelování časových řad ve smyslu
kombinatorického testování všech příznaků za účely konstrukce výstupního
příznakového kauzálního grafu (resp. kauzální matice, viz
sekce~\ref{sec:hybridni_detekce}). Takové řešení by ale pro poměrně velký počet
fyziologických příznaků bylo extrémně výpočetně náročné, a proto bylo dále
uznáno za nevhodné.

Řešením se zde naskytla regrese, kterou lze využít pro identifikaci podmnožiny
příznaků, na které je daný příznak podmíněně závislý. To vychází z faktu, že
nejlepší regresor pro danou proměnnou s nejmenší kvadratickou chybou bude mít
teoreticky nenulové koeficienty pouze pro proměnné v okolí\footnote{Statisticky
    \enquote{okolí} proměnné implikuje podmnožinu proměnných, které s ní úzce
    souvisejí.}~\cite{Schindler2013,Arnold2007}. Pro tento regresní problém byl
zvolen Lasso algoritmus, jež je uveden do souvislosti v následující sekci.

\subsection{Lasso regrese}
\label{subsec:lasso}
\gls{Lasso} (Least Absolute Shrinkage and Selection Operator) je široce
používaná technika lineární regrese pro výběr a regularizaci proměnných využitím
$\ell_1$ penalizačního členu. Formálně, výstup $\vec{w}$ minimalizuje součet
průměrné kvadratické chyby regrese pro $y$:
\begin{equation}
    \vec{w}=\arg \min \frac{1}{n} \sum_{(\vec{x}, y) \in X}|\vec{w} \cdot \vec{x}-y|^2+\lambda\|\vec{w}\|_1
\end{equation}
kde $X$ je vstupní příznak, $n$ je počet vzorků v $X$ a $\lambda$ je penalizační
člen určující míru regularizace koeficientů. S rostoucí hodnotou $\lambda$ se
více koeficientů smršťuje směrem k nule, což vede k řídkému modelu s menším
počtem prediktorů a naopak~\cite{Tibshirani1996}. Ve smyslu tvorby kauzálních
matic poskytuje Lasso množinu časových proměnných, které při regresi $y_t$ podle
zpožděných proměnných $x_{t'}$, kde $t'=\{t-T,...,t -1\}$ pro všechna $x \in X$,
nabývají právě nenulového koeficientu Grangerovy kauzality.

Nicméně Bahadori a Liu dokázali v~\cite{Bahadori2013}, že Grangerova kauzalita
je v rámci použití vícerozměrných dat inkonzistentní a není dobře schopna
zachytit nelineární vztahy nebo složité struktury závislostí. Vzhledem k tomu,
že data využívané v této práci jsou vícerozměrná a reprezentují biosignály, v
rámci kterých mohou existovat právě nelineární vztahy nebo komplexní
hierarchické relace, byl využit přístup Kopula-Granger. S využitím Lasso ukázali
v~\cite{Bahadori2013} jeho konzistenci na vícerozměrných datech i jeho schopnost
efektivně zachytit nelinearitu v datech (viz
sekce~\ref{subsec:kauzalni_matice}).

\subsection{Teorie kopulí}
\label{subsec:teorie_kopul}
Vzhledem k využití Kopula-Granger přístupu pro tvorbu vícerozměrných kauzálních
matic je žádoucí stručně představit kopula funkce. V teorii pravděpodobnosti a
statistice je kopula pravděpodobnostní distribuční funkcí, jež popisuje
závislost mezi jednotlivými marginálními distribucemi a poskytuje způsob jak
modelovat právě společnou distribuci náhodných veličin bez specifikace samotných
distribucí těchto veličin. Základem teorie kopulí je Sklarův teorém, který říká,
že jakákoli více-dimenzionální distribuce může být zapsána jako kopula
aplikovaná na její marginální distribuce:

\begin{theorem}[Sklarův teorém]
    \label{theorem:sklar}
    Nechť $F$ je vícerozměrná distribuční funkce s marginálními distribucemi
    $F_1, F_2, \ldots, F_n$. Pak existuje kopula $C$ taková, že:
    \begin{equation}
        F\left(x_1, x_2, \ldots, x_n\right)=C\left(F_1\left(x_1\right), F_2\left(x_2\right), \ldots, F_n\left(x_n\right)\right)
    \end{equation}
    Kopula $C$ je jednoznačná, pokud marginální distribuce $F_1, F_2, \ldots, F_n$ jsou spojité.
\end{theorem}

Kopula-Granger model, ve kterém jsou v této práci kopula funkcí mapovány
marginální distribuce fyziologických událostí do kopula prostoru je popsán v
sekci~\ref{subsec:kauzalni_matice}. Pro shrnutí vychází tento model z
následujících kroků~\cite{Guy2016}:
\begin{enumerate}
    \item Nalezení empirického marginálního rozdělení pro fyziologickou událost
          $\hat{F_i}$.
    \item Mapování pozorování do kopula prostoru:
          $\hat{f}_i\left(x_t^i\right)=\hat{\mu}_i+\hat{\sigma}_i.
          \Phi^{-1}\left(\hat{F}_i\left(x_t^i\right)\right)$.
    \item Nalezení Grangerovy kauzality v rámci $\hat{f}_i\left(x_t^i\right)$.
\end{enumerate}
přičemž je dále brán v potaz Winsorizovaný\footnote{Winsorizace nebo Winsorova
transformace je transformace statistických dat omezením extrémních hodnot, aby
se snížil vliv případných odlehlých hodnot} odhad použité distribuční funkce,
podle~\cite{Bahadori2013}, aby se zabránilo velkým číslům
$\Phi^{-1}\left(0^{+}\right)$\footnote{$\Phi^{-1}$ je inverzní kumulativní
distribuční funkce standardního normálního rozdělení.} and
$\Phi^{-1}\left(1^{-}\right)$:
\begin{equation}
    \tilde{F}_j= \begin{cases}\delta_n, & \text { if } \hat{F}\left(x^j\right)<\delta_n \\ \hat{F}\left(x^j\right) & \text { if } \delta_n \leq \hat{F}\left(x^j\right)<1-\delta_n \\ \left(1-\delta_n\right) & \text { if } \hat{F}\left(x^j\right)>1-\delta_n .\end{cases}
\end{equation}

\subsection{Metriky hodnocení v strojovém učení}
\label{subsec:ml_metriky}



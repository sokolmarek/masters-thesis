Obory umělé inteligence, jako strojové učení nebo neuronové sítě, často vyžadují
v reálných podmínkách pečlivou přípravu a předzpracování dat nebo sestavení a
trénování modelů. V dnešní době však existuje velké množství nástrojů a
knihoven, které tyto kroky implementují a značně tak zvyšují efektivitu vývoje
patřičných aplikací. Tato kapitola popisuje zásadní nástroje použité pro účely
této práce.

\subsection{Python a R}
\label{subsec:python_r}
Mezi nejpopulárnější open-source programovací jazyky v oblasti strojového učení
a data science, které byly zároveň použity v této práci, patří
Python\footnote{https://www.python.org} a R\footnote{https://www.r-project.org}.
Pro předzpracování dat, strojové učení a neuronové sítě byl použit
Python 3.7 s využitím platformy Google Colab.

Explorační a statistická analýza dat byla realizována prostřednictvím jazyka R
(verze 4.2.1, Funny-Looking Kid) na laptopu \textit{HP Spectre x360} s
procesorem \textit{i7-8705G}, 32~GB DDR4 RAM a grafickou kartou \textit{RX Vega
M GL}. I přestože R není na rozdíl od Pythonu univerzálním vysokoúrovňovým
programovacím jazykem a využívá se především pro statistické modelování, je díky
bohaté komunitě a velkému množství knihoven nedílnou součástí oblasti strojového
učení a data science. 

\subsection{Google Colab a Jupyter Notebook}
\label{subsec:jupyter_colab}
Na základě velkého objemu dat ke zpracování bylo využito platformy Google
Colab\footnote{https://colab.research.google.com}. Jedná se o cloudové
interaktivní výpočetní prostředí, které běží na virtuálním stroji a umožňuje
vzdálené spuštění kódu s využitím prostředků jako \textit{NVIDIA Tesla
V100/P100} s 24~GB VRAM. Jinými slovy jde o hostovanou webovou aplikaci jménem
Jupyter Notebook\footnote{https://jupyter.org}, která umožňuje vytvářet a sdílet
dokumenty (zápisníky). Tyto dokumenty jsou rozděleny do buněk, které lze
spouštět v libovolném pořadí (live kód), což zajišťuje efektivnější
prototypování.

\subsection{Neurokit}
\label{subsec:neurokit}
Knihovna Neurokit2~\cite{Makowski2021neurokit} poskytuje pokročilé metody pro
zpracování a vizualizaci biosignálu. Jednotlivé metody zároveň nabízejí možnost
si vybrat z mnoha implementovaných algoritmů, například pro detekci QRS
komplexu. V této práci byla knihovna použita pro předzpracování respirační,
elektrodermální a srdeční aktivity včetně zpracování HRV.

\subsection{Tidyverse a Easystats}
\label{subsec:tidyverse_easystats}
Knihovny tidyverse~\cite{tidyverse} a easystats~\cite{easystats} rozšiřují jazyk
R o mnoho funkcionalit primárně pro potřeby statistického modelování a
strojového učení. Usnadňují a zrychlují proces tvorby modelů díky dobře
zdokumentovanému ekosystému balíčků. V této práci sloužili knihovny ke
statistické analýze velkého souboru dat. 

\subsection{Scikit-learn, TensorFlow a Keras}
\label{subsec:scitkit_tensor_keras}
Scikit-learn~\cite{sklearn_api} je balíček jazyka Python pro prediktivní analýzu
dat a strojové učení, který byl v této práci použit pro extrakci a normalizaci
příznaků. Dále pro porovnávání, validaci a výběr parametrů a modelů.

% TensorFlow is an open-source framework developed at Google for machine learning
% applications. Its main focus is on defining the architecture and training of
% deep neural networks. It is highly optimized for the execution of low level
% tensor operations on CPU, GPU, or TPU. 

% Keras is a high-level API that acts as an interface for the TensorFlow
% framework. It enables faster prototyping of ANNs by providing abstractions and
% building blocks for developing the models. It also provides the implementation
% of several popular CNN architectures along with their weights, making transfer
% learning more accessible. The simplest way of defining Keras models is by using
% the Sequential model API, which is essentially a linear stack of defined layers.
% The alternative is adopting the Keras functional API, which allows for building
% arbitrary graphs of layers with multiple inputs and outputs or using residual
% skipping connections.

\subsection{InfluxDB}
\label{subsec:influx}
InfluxDB\footnote{https://www.influxdata.com} je open-source platforma
poskytující databázi pro časové řady. Zahrnuje rozhraní (API) pro standardní
databázové dotazy. Součástí je i grafické uživatelské rozhraní (GUI) s
modulárními uživatelskými panely pro monitorování dat v reálném čase. Tato
platforma (InfluxDB OSS 2.4) byla využita v rámci experimentální části práce k
uchovávání a vizualizaci dat.


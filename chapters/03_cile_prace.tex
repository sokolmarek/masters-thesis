Hlavním cílem diplomové práce je návrh a realizace metod pro hodnocení
kognitivní zátěže z biosignálů současně s vyhodnocením vlivu extrémního
prostředí na její projevy. Hodnocenými biosignály jsou konkrétně elektrická
srdeční aktivita, respirační aktivita a elektrodermální aktivita.

Dalším cílem práce je návrh vhodného protokolu studie pro měření v podmínkách
analogových misí studujících vliv izolace člověka, který bude vycházet ze
současných publikovaných datasetů sledujících vliv kognitivní zátěže na
fyziologické signály.
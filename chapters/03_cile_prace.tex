Hlavním cílem diplomové práce je návrh a realizace metod pro hodnocení
kognitivní zátěže z biosignálů současně s vyhodnocením vlivu extrémního
prostředí na její projevy. Hodnocenými biosignály jsou konkrétně elektrická
srdeční aktivita, respirační aktivita a elektrodermální aktivita. Extrémním
prostředím je v kontextu diplomové práce myšlena vesmírná analogová mise, během
které proběhl sběr dat.

% Hlavním cílem diplomové práce je návrh a realizace metod pro hodnocení
% kognitivní zátěže z biosignálů současně s vyhodnocením vlivu extrémního
% prostředí na její projevy. Hodnocenými biosignály jsou konkrétně elektrická
% srdeční aktivita, respirační aktivita a elektrodermální aktivita. Extrémním
% prostředím je v kontextu diplomové práce myšlena vesmírná analogová mise.

Cílem diplomové práce je vyhodnocení vlivu extrémního prostředí na projevy
kognitivní zátěže v biosignálech analýzou dat získaných v průběhu simulované
vesmírné mise, studující vliv izolace na člověka. Dále ma být hodnocen vliv úloh
navozujících zvýšenou kognitivní zátěž na elektrickou srdeční, dechovou a
elektrodermální aktivitu společně s využitím časových, frekvenčních a
nelineárních parametrů sledovaných signálů. Výsledky hodnocení se srovnají v
rámci prostředí s vysokým a standardním atmosférickým tlakem. Pro účely práce je
tedy nezbytný návrh a realizace metod pro hodnocení kognitivní zátěže.

% Vyhodnoťte vliv extrémního prostředí na projevy kognitivní zátěže v
% biosignálech. Analyzujte data získaná v průběhu simulované vesmírné mise,
% studující vliv izolace na člověka. Vyhodnoťte vliv úloh navozujících zvýšenou
% kognitivní zátěž na EKG, dechovou frekvenci a kožní vodivost. Pro analýzu
% použijte časové, frekvenční a nelineární parametry sledovaných signálů.
% Srovnejte projevy kognitivní zátěže v prostředí s vysokým a standardním
% atmosférickým tlakem.
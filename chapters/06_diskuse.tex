Diplomová práce se zabývá hodnocením kognitivní zátěže z periferních biosignálů
získaných v průběhu vesmírné analogové mise studující vliv izolace na člověka.
Měřenými biosignály byly zejména elektrická srdeční, respirační a
elektrodermální aktivita. Pro účely hodnocení kognitivní zátěže představuje tato
práce nový multimodální přístup. Experimentální část, realizované dílčí části
práce a jejich výsledky jsou rozebrány v této kapitole na několika úrovních.

\paragraph{Analogová vesmírná mise}
Extrémním prostředím je v kontextu diplomové práce myšlen vesmír vzhledem k
charakteru experimentální části práce, která je popsána v
sekci~\ref{sec:mise_diana}. Šesti členná posádka podstoupila po dobu osmi dní
experiment v podobě analogové vesmírné mise, která simulovala přistání na
měsíci. Posádka byla rozdělena na dvě tříčlenné skupiny, mateřskou loď a
přistávací modul. Členové přistávacího modulu absolvovaly experiment v
hyperbarickém saturačním prostředí ve formě podvodní laboratoře Hydronaut H03
DeepLab (viz sekce~\ref{subsubsec:h03_deeplab}), kde žili a plnili celodenní
úkoly po celou dobu mise. Úkoly zahrnovaly extravehikulární aktivity,
nepřetržitou komunikaci s mateřskou lodí a řídícím střediskem, provádění výzkumu
a další činnosti, jež byly analogické operačním požadavkům a rizikům spojeným s
kosmickými lety a misemi. Mise proběhla bez významných komplikací.

Zpětně lze říci, že analogová mise DIANA vykazovala vysokou míru provozní a
infrastrukturní robustnosti a použitelnosti. To plyne primárně z rozsáhlých
zkušeností týmu projektu Hydronaut a z integrace vybraných odborníků a studentů
do jednotlivých kompartmentů mise. Experimentu také předcházela pečlivá a včasná
příprava technických aspektů včetně specializovaných IT systémů pro její
podporu. Mise DIANA přinesla bohaté zkušenosti, které je třeba zvážit a zahrnout
do budoucích experimentů. Vzhledem k tomu, že jednou z myšlenek projektu
Hydronaut je využití výzkumného prostředí Hydronaut H03 DeepLab pro výcvik
budoucích astronautů, přichází v úvahu srovnání s \gls{NASA} \gls{NEEMO}
podmořskou výzkumnou laboratoří Aquarius, kde každoročně probíhají analogové
vesmírné mise. Experiment probíhal pod režií \gls{FF UPOL} a tudíž se orientoval
primárně na psychologický výzkum. Pro potřeby diplomové práce nebo širšího
biomedicínského výzkumu tak nebylo experimentální nastavení příliš vhodné. To
hlavně z hlediska nezahrnutí dostatečného sběru biomedicínských dat před a po
experimentu samotném. Tím pádem je možná analýza pouze dat získaných v průběhu
bez možnosti porovnání se stavem před a po saturačním ponoru. Na rozdíl ale od
\gls{NASA} \gls{NEEMO} experimentů~\cite{koutnik2021neemo}, kde měří vybrané
veličiny pouze v určité dny a časové úseky, byla měřena biomedicínská data každý
den nepřetržitě. To může poskytovat jistou výhodu v náhledu do vývoje fyziologie
v průběhu adaptace organismu na extrémní prostředí. Avšak takové kvantum
získaných dat je na druhou stranu velmi časově náročné zpracovat a anotovat.
Proto anotace v rámci diplomové práce nebyla realizována a pro vyhodnocení vlivu
úloh navozujících zvýšenou kognitivní zátěž bylo postačující označení úseků, ve
kterých byli jedinci podrobeni baterii testů (viz
sekce~\ref{subsubsec:neuro_testy}). K anotaci budou v budoucnu včetně
harmonogramu sloužit kamerové záznamy, které byly pořízeny v průběhu celé mise.
Naskytuje se tak možnost využití kamerových záznamů pro tvorbu anotačních
nástrojů, které by mohli do jisté míry automatizovat a usnadnit tento proces.
Dále by bylo vhodné zvážit rozšíření experimentálního protokolu o měření dalších
biomedicínských dat, která využívá v experimentech \gls{NASA} pro zkoumání
adaptace člověka na vícedenní hyperbarickou saturaci~\cite{koutnik2021neemo}.
Jedná se například o monitorování periferní a cerebrovaskulární hemodynamiky,
tělesné kompozice a teploty nebo kvality spánku.

V neposlední řadě je třeba poukázat na problémy se sběrem biomedicínských dat v
průběhu mise. Umístění senzorů lze vidět v
sekci~\ref{subsubsec:mereni_biosignalu} na Obr.~\ref{fig:sensors}. Pro měření
elektrické srdeční aktivity byly používány nalepovací elektrody, které poměrně
často vysychaly a bylo tak nutné dodatečně používat vodivý gel a další fixaci
pomocí kineziologických pásek. Vhodnou alternativou by mohlo být použití
textilních elektrod, jejichž vývoj je v posledních letech velmi dynamický.
Zároveň přináší výhodu opakovatelného použití, mytí a biokompatibility. Některé
studie dokonce ukazují vyšší odolnost vůči pohybovým artefaktům než je tomu tak
u standardních Ag/AgCl elektrod~\cite{Nigusse2020,Nigusse2021}. Dále nastával
problém s nežádoucím posunem dechového pásu, který se ale vyřešil použitím
kšand. Měření kožní vodivosti mimo laboratorní podmínky je velmi náročné samo o
sobě vzhledem k velké citlivosti na pohybové artefakty. Pro potřeby experimentu
nepřicházeli běžné umístění elektrod na prsty rukou v úvahu. Bylo tak zvoleno
alternativní místo na základě studie~\cite{Janssen2012}. Elektrody byly lepeny v
úrovni mečovitého výběžku sterna v oblastech mezi sternální a parasternální
čárou na levé i pravé straně. Během spánku se také stávalo, že si jedinec
přilehl či odpojil kabely z měřícího zařízení nebo došlo k jeho vybití. Vyskytlo
se tak dilema, zdali je vhodné jedince budit nebo přečkat do jeho probuzení.
Veškeré tyto zkušenosti by měly být využity a zohledněny v dalších iteracích
vývoje měřícího systému pro účely analogových vesmírných misí.

\paragraph{Zpracování dat}
Metodika, která byla aplikována pro zpracování dat z mise DIANA je popsána v
sekci~\ref{sec:zpracovani_dat_diana}. Samotnému zpracování biosignálů předcházel
výběr vhodné metody předzpracování signálu a detekce komponentů. To se týkalo
především elektrické srdeční aktivity vzhledem k hojnému počtu existujících
metod pro její zpracování. Neexistuje však žádný konsensus ohledně toho, kterou
z nich použít nebo která je nejlepší pro danou aplikaci. Je pravděpodobné, že
všechny mají nějaké silné a slabé stránky. Běžně se pro účely hodnocení těchto
metod využívají datasety, ve kterých jsou expertem anotovány jednotlivé
komponenty, v tomto případě R vlny. Hodnocení tedy logicky probíhá na základě
odchylky detekované R vlny od referenční anotované R vlny. Poslední
studie~\cite{Porr2019}, která se věnuje rozsáhlejšímu porovnání populárních
metod, využívá pro hodnocení tradiční metriky, jako je senzitivita a přesnost.
Zároveň se vyhrazuje pouze na jednu databázi, což může být zavádějící z hlediska
generalizace. Studie~\cite{Porr2019} ve výsledku prezentuje jako nejlepší
metodu, se kterou přišli Elgendi et al.~\cite{Elgendi2010}. V této práci bylo
porovnání metod provedeno využitím statistického modelování v rámci několika
datasetů (viz sekce~\ref{subsubsec:vyberqrs}). Výsledky prezentované v
Tabulce~\ref{tab:qrs} ukazují, že nejlepších výsledků dosáhla metoda, kterou
představili autoři Kalidas et al.~\cite{kalidas2017}. Rozdíly mohou být
způsobeny právě použitím více databází a modelování s ohledem na jejich
hierarchickou strukturu pomocí lineárních smíšených modelů. I přestože se jedná
o poměrně rozsáhlý způsob srovnání, tak se postupem práce ukázalo, že výběr
nejlepší metody takovým způsobem nemusí být úplně vhodný. To se týká především
datasetů, které byly použity pro strojové učení (viz sekce~\ref{sec:datasety}).
V rámci datasetu CLAS vybraná metoda předzpracování nefungovala u značného
množství záznamů zdaleka tak dobře, jako u datasetu WESAD. To se dále
projevovalo přeučením modelů, které zahrnovaly dataset CLAS.

V návaznosti na problematiku zvolení vhodných metod pro zpracování biosignálů se
dále vyskytl poměrně zásadní problém v rámci hodnocení kvality signálu. To úzce
souvisí s výběrem sledovaných veličin v této práci. Představa manuální inspekce
kvality signálů u záznamů z mise se jevila jako nereálná (8 dní záznamů), a
proto přicházelo v úvahu tento proces do určité míry automatizovat. V rámci
literatury se vyskytuje poměrně velké množství metod pro evaluaci kvality EKG
signálu, a to i veřejně dostupných~\cite{Bijl2022,Satija2018}. Postup hodnocení
uplatněný v této práci je popsán v sekci~\ref{subsec:prezpracovani_segmentu}.
Problém však nastal u hodnocení kvality respirační a elektrodermální aktivity,
pro jejichž hodnocení nejsou k dispozici žádné veřejně dostupné nástroje. Z
časových důvodu nebyl ani realizován žádný vlastní nástroj. Data tohoto
charakteru proto byla použita pouze v rámci úloh strojového učení, nikoliv pro
část práce, která se věnuje statistické analýze spojené s hodnocením \gls{NPF}
změn v průběhu mise.

\paragraph{Analýza dat mise}
Logistika mise a bezpečnost přirozeně omezily počet subjektů v rámci měření dat
s analýzou a i ve schopnosti vyvozovat statisticky významné závěry. Zároveň
pouhé zastoupení mužů omezuje extrapolaci zjištění mezi pohlavími. Ohled je
třeba brát také na porovnání skupin, jelikož se u skupiny přistávacího modulu
jednalo o trénované certifikované potápěče. Popis posádky mise společně s
metodikou jejího výběru se nachází v kapitole~\ref{sec:mise_diana}. K analýze
dat bylo včetně grafických metod, jako jsou bodové rozsahy nebo krabicové grafy,
využito lineárních smíšených modelů (viz sekce~\ref{subsec:tvorba_modelů}).

Ačkoli nelze hovořit o statisticky signifikantních rozdílech, tyto modely
poskytují určitý pohodlný vhled do objemných dat mise. Pro odhady smíšených
modelů má zvláštní význam to, že pro praktické účely musí existovat přiměřený
počet úrovní náhodných efektů~\cite{Clark2015}. To je v případě experimentu s
šesti subjekty splněno~\cite{Gomes2022}. Následně bylo možné využít odhady
smíšených modelů k vytvoření obrazu o průběhu změn sledovaných veličin na úrovní
dnů, skupin a i jednotlivých subjektů. Pokud se vrátíme k otázce statistické
významnosti, je třeba si uvědomit, že je nutno se orientovat dle síly modelu,
jež vypovídá o pravděpodobnosti odhalení účinku zájmu, pokud skutečně existuje.
Stejně jako v případě t-testů, kde byla prokázána přijatelná statistická síla v
rámci extrémně malých velikostí vzorků~\cite{Winter2013}, je možné takových
charakteristik dosáhnout i u \gls{LMM}~\cite{Muth2016}. To se mimo jiné odvíjí
od designu samotného modelu. Není tedy namístě při malém počtu subjektů ihned
zavrhovat významné výsledky modelu. V případě modelů navržených v této práci
byla analyzována i jejich síla (viz sekce~\ref{subsec:tvorba_modelů}). Síla
modelů však podle předpokladů vycházela v rozmezí 45--65 \%, což je
pravděpodobně způsobeno jeho složitějším designem společně s dolní hranicí počtu
smíšených efektů. V této práci nebyly vysloveny žádné konkrétní hypotézy, tak
tomu ale nemusí být v případě budoucích experimentů, které se budou potýkat s
podobnými omezeními. \gls{NASA} například u svých \gls{NEEMO} experimentů, kde
se počet subjektů pohybuje kolem deseti, využívá tradičních metod (t-testy,
korelace, ANOVA, aj.) nebo jen popisné
statistiky~\cite{koutnik2021neemo,Chappell2013}. Zdůrazňuje i limitaci
schopnosti vyvozovat relevantní statistické závěry na základě takových
dat~\cite{koutnik2021neemo}. \gls{NASA} však provádí tyto experimenty pravidelně
každý rok a zároveň má k dispozici data před a po vesmírných letech, která
nejsou veřejnosti dostupná. Vzniká tak každopádně prostor pro návrh metodiky
statistického \enquote{ad-hoc} hodnocení \gls{ICE} experimentů s ohledem na
jejich stěžejní specifika.

\paragraph{Srdeční a autonomní regulace}
U subjektů Lander1, MotherShip1 a MotherShip3 nedocházelo v průběhu mise u
hodnoty SD2 k výrazným změnám, až na třetí den kdy došlo u MotherShip1 k
viditelnému poklesu a stejně jako poslední den u Lander1. U Lander3 došlo k
poklesu třetí den a následně k růstu hodnoty SD2 až do konce mise. Lander4
vykazoval růst parametru do třetího dne, po kterém SD2 také klesal do konce. V
rámci MotherShip2 si parametr SD2 udržoval klesající trend v průběhu mise.
Vývoje parametru SD2 prezentuje Obr.~\ref{fig:results_pointrange_sd2}. Průměrné
hodnoty parametru SD2 pro vybrané dny jsou uvedeny v
Tabulce~\ref{tab:daily_params}.

Parametr \gls{RMSSD} nevykazoval výrazné změny oproti vývojům parametru SD2.
Pouze u Lander1 došlo k viditelnějšímu poklesu poslední den, kterému předcházel
vyšší vzrůst. Vývoje parametru RMSSD jsou uvedeny na
Obr.~\ref{fig:results_pointrange_rmssd}. Průměrné hodnoty parametru RMSSD
společně se směrodatnými odchylkami pro vybrané dny jsou uvedeny v
Tabulce~\ref{tab:daily_params}.

U pNN50 byly pozorovány výrazně stabilnější vývoje něž u předchozích metrik.
Denní výkyvy byly u všech subjektů menší a trendy byly zachovány. Mediánové
průběhy pNN50 lze vidět na Obr.~\ref{fig:results_pointrange_rmssd} a průměrné
hodnoty pro první, čtvrtý a osmý den jsou uvedeny v
Tabulce~\ref{tab:daily_params}.

Proměnná HF zachovává podobný průběh s předešlými parametry pouze pro jedince
MotherShip3. MotherShip1 a MotherShip2 vykazují klesající charakter parametru po
prvním dnu. U Lander1 drží stabilní hodnoty kromě náhlého vzrůstu třetí den.
Lander3 má klesající vývoj do čtvrtého dne, od kterého dochází k růstu parametru
až do konce mise. Lander4 má stejný vývoj jako Lander3 jen klesání lze pozorovat
až do 5 dne. Průběhy HF je možné pozorovat na
Obr.~\ref{fig:results_pointrange_rmssd} a průměrné hodnoty parametru pro
začátek, střed a konec mise jsou uvedeny v Tabulce~\ref{tab:daily_params}.

U všech sledovaných veličin dochází průměrně napříč dny k jejich poklesu,
přičemž jsou tyto poklesy menší u skupiny přistávacího modulu než u mateřské
lodi (viz Tabulky~\ref{tab:lmm_rmssd_SD2} a \ref{tab:lmm_pnn50_hf}). Velikost
rozdílů je však relativně malá. Odhady fixních efektů referovaných tabulek jsou
vyneseny na Obr.~\ref{fig:results_lmm_coefs1} a \ref{fig:results_lmm_coefs2}.
Vizualizace \gls{LMM} je pro SD2 a RMSSD na Obr.~\ref{fig:results_lmm_fit1}
a.~\ref{fig:results_lmm_fit2}. Pro pNN50 a HF na Obr.~\ref{fig:results_lmm_fit3}
a.~\ref{fig:results_lmm_fit4}.

Po saturačním ponoru lze u skupiny přistávacího modulu pozorovat v rámci
bodových rozsahů jistý trend. Například u HF má tento trend v prvních dnech
klesající charakter a následně do konce mise je rostoucí. Nabízí se tak myšlenka
toho, že po prvním vystavení hyperbarickému prostředí by mohlo dojít k akutní
stresové reakci, která by způsobila snížení HF v důsledku zvýšené aktivity
sympatiku. Následně po aklimatizaci subjektu na hyperbarické podmínky může dojít
k obnovení nebo zvýšení HF, protože převládne parasympatický
systém~\cite{Lafere2021,Lund1999,Lund2000,Ma2022}. Pozorované změny jsou ale
velmi malé (nevýznamné), což díky měřítku nemusí ihned vyplývat z bodových
rozsahů (viz
Obr.~\ref{fig:results_pointrange_sd2}--~\ref{fig:results_pointrange_hf}).
Vypovídají o tom spíše statistické modely v sekci~\ref{sec:vysledky_lmm}. To
platí pro všechny parametry. Minimální výkyvy na reakci prostředí jsou v souladu
s faktem, že se jedná o trénované potápěče. U skupiny mateřské lodi lze
pozorovat klesající charakter sledovaných veličin v průběhu mise. Rozdíly ale
opět nejsou významné. Vzhledem k chybějícím datům před a po experimentu nelze
učinit další závěry o vlivu hyperbarického prostředí.

Obecně byla četnými studiemi prokázaná zvýšená parasympatická a potlačená
sympatická modulace~\cite{Lafere2021,Lund1999,Lund2000,Ma2022}. Současné
poznatky tak v případě prostředí hyperbarické saturace vypovídají o stimulaci
vagové modulace, snížení HR, zvýšení variability R--R intervalů a zvýšení
HF~\cite{Goyal2023}. Další studie zároveň poukazují na zvýšení parametrů RMSSD,
pNN50 nebo SD2~\cite{koutnik2021neemo,Lafere2021,Lund2000,Ma2022}. Zvýšená
vagová modulace v hyperbarickém prostředí je přisuzována jak zvýšenému obsahu
$\text{O}_2$, tak zvýšenému tlaku (nezávisle na
$\text{O}_2$)~\cite{koutnik2021neemo,Lund1999}. Mechanismy pro tuto vagovou
odezvu jsou připisovány vagové aktivitě kompenzující nedostatek kyslíku,
vazokonstrikci (baroreflex), zvýšenou srdeční účinností v důsledku vystavení
zvýšenému $\text{PPO}_2$ v srdeční tkáni, prodloužením Q--T intervalu nebo
inhibicí karotického chemoreflexu~\cite{koutnik2021neemo,Goyal2023,Paula2019}.
Otázkou zůstává, zdali je možné vlivem hyperbarického prostředí
\enquote{saturovat} HRV parametry, které se běžně využívají k hodnocení
kognitivní zátěže, a učinit tak jejich interpretaci pro tuto úlohu stěžejní.

\paragraph{Neuropsychofyziologická regulace}
Během vesmírné analogové mise DIANA podstoupil každý subjekt dvě až tří sezení,
během kterých byl podroben baterii kognitivních testů (viz
sekce~\ref{subsubsec:neuro_testy}). Vypočtené sledované veličiny pro jednotlivá
sezení jsou pro každého člena posádky vyneseny na
Obr.~\ref{fig:results_boxplot_SD2_tests} až \ref{fig:results_boxplot_hf_tests} v
podobě krabicových grafů. Průměrné hodnoty parametrů pro každé sezení jsou
uvedeny v Tabulce~\ref{tab:tests_params_lander}
a~\ref{tab:tests_params_mothership}. U subjektů Lander3 a MotherShip1 lze každé
sezení sledovat vzrůstající trend parametrů SD2, RMSSD a pNN50. Pouze u HF
klesající. Studie~\cite{Castaldo2019,Pham2021,Ishaque2021} uvádí, že v reakci na
kognitivní zátěž dochází ke snížení sledovaných parametrů. To vyplývá také z
modelu neuroviscerální integrace, jež je popsán v
sekci~\ref{sec:neurovisceralni_integrace}. U těchto subjektů je tedy možné
usuzovat zlepšení kognitivní výkonnosti v průběhu mise. To však není v souladu s
průběhem složky HF, která podle~\cite{Forte2019} vypovídá o adaptaci jedince na
měnící se požadavky prostředí. Její redukce by mohla naznačovat nedostatečnou
schopnost jedince flexibilně reagovat na enviromentální nároky a tedy vypovídat
o jistém konfliktu na úrovni kognitivní regulace v rámci \gls{ICE}
prostředí~\cite{Forte2019,Ayres2021}. Naopak u subjektů MotherShip2 a
MotherShip3 bylo pozorováno snížení všech parametrů mezi sezeními, a tudíž by
šlo hovořit o zhoršení kognitivní výkonnosti v průběhu mise. U subjektů Lander1
a Lander4 dochází k minimálním změnám, které naznačují, že ke zlepšení došlo
pouze u prostředního sezení (časově uprostřed mise). Sledované veličiny,
variabilita srdečního rytmu a její regulace byly popsány v
kapitole~\ref{sec:hrv}.	

Při vyhodnocení vlivu úloh navozujících zvýšenou kognitivní zátěž pomocí
sledovaných veličin, neboli časových, frekvenčních a nelineárních parametrů,
nastává však zásadní problém z hlediska interpretace. Jak již bylo řečeno v
úvodu práce, veškeré kognitivní funkce se promítají do fyziologie člověka.
Sekce~\ref{subsec:hrv_indices} poukazuje na pragmatické problémy, jako je
nejasnost asociace HRV ukazatelů s fyziologickými mechanismy nebo podobnosti a
překryvy mezi množstvím z těchto metrik. Další otázkou je, zdali při dlouhodobém
pobytu lidského organismu ve vesmíru lze z hlediska smyslu a podstaty
předpokládat stálost těchto parametrů pro účely hodnocení \gls{CL}. V ideálním
případě by bylo vhodné realizovat systematické porovnání vypočtených parametrů
pro každého jedince a inkorporovat metody, například z oborů analýzy dat genové
exprese, které se zabývají klasifikací velkého souboru objektů podle přirozených
vztahů. Jednou z takových metod je konsenzuální shlukování~\cite{Monti2003},
které vychází z představy, že provedení nekonečného počtu shlukovacích algoritmů
odhalí \enquote{skutečné} shluky. V rámci aplikace této metody na soubor
fyziologických parametrů by bylo následně možné z pravděpodobnostní perspektivy
nahlédnout nejen na asociace a vztahy mezi jednotlivými parametry, ale také na
jejich fyziologický smysl. Na druhou stranu se jedná o poměrně časově a datově
náročné řešení včetně nezbytnosti zahrnutí \gls{NVI} modelu.

\paragraph{Hodnocení kognitivní zátěže}
Pro potřeby této práce byl navržen nový multimodální přístup k hodnocení
kognitivní zátěže z periferních biosignálu pojmenovaný \textit{Hodnocení
kognitivní zátěže využitím vícerozměrných časoprostorových kauzálních vzorů}.
Metoda je popsána v samostatné kapitole~\ref{sec:hybridni_detekce}. Celkem bylo
natrénováno $3 \times 6$ modelů na datasetech, jejichž popis je uveden v
sekci~\ref{sec:datasety}. Bylo experimentováno i s augmentací dat
(viz~\ref{subsec:augmentace_dat}). Výsledky trénování na datasetu WESAD jsou
uvedeny v sekci~\ref{subsec:wesad_models}. Největší validační přesnost 98,25 \%
byla dosažena u augmentovaného modelu WESAD-5s50 (viz
Tabulka~\ref{tab:wesad_eval}). Natrénované modely pro dataset CLAS jsou
prezentovány v sekci~\ref{subsec:clas_models}, přičemž nejvyšší testovací
přesnosti 99,5 \% dosáhl augmentovaný model CLAS-5s50 (viz
Tabulka~\ref{tab:clas_eval}). Poslední sada modelů byla natrénována na
sjednocených datasetech WESAD a CLAS. Tyto modely byly pojmenovány jako
\enquote{Merged} a jejich výsledky se nacházejí v sekci
\ref{subsec:merged_models}. Nejvyšší přesnosti 99,55 \% dosáhl augmentovaný
model Merged-5s50 (viz Tabulka~\ref{tab:merged}). 

Natrénované modely byly následně otestovány na datech z mise DIANA. Došlo k
zjištění, že u modelu CLAS (tedy i Merged) došlo do jisté míry pravděpodobně k
přeučení vzhledem ke špatné generalizaci. To je přisuzováno především nevhodnému
předzpracování biosignálu, jak již vyplývá z dřívější části diskuze. Pro
testovací účely byl tedy primárně zvolen model WESAD-5s. Výsledek testování lze
vidět v sekci~\ref{subsec:ml_diana_data_test} na Obr.~\ref{fig:hydro_test}. V
případě tohoto výsledku jsou zájmem primárně vzniklé trendy, které mají v drtivé
většině případů inverzní charakter v porovnání s výsledky \gls{NPF} stimulace
(modré čárkované trendy na
Obr.~\ref{fig:results_boxplot_SD2_tests}--~\ref{fig:results_boxplot_hf_tests}).
To je v souladu s očekáváním a tvrzeními o vlivu úloh navozujících zvýšenou
kognitivní zátěž během mise v předešlých odstavcích. Osa $y$ ale v této podobě
není příliš intuitivní. Proto byl vytvořen Obr.~\ref{fig:hydro_test_bar}, který
prezentuje relativní četnosti výskytu pravděpodobností \gls{CL} vyšších než
50~\% pro jednotlivá sezení. Zde je již patrnější, během kterých sezení a v jaké
míře byla zaznamenána kognitivní zátěž. Pro další a podrobnější testování
generalizace je však nutné data detailněji anotovat. Protože byly cílem primárně
5s modely, je nutné v budoucích experimentech otestovat například i realizované
1s modely. Předběžné testy nevykazovaly přeučení 1s modelů (WESAD), což by mohlo
znamenat, že lze \gls{CL} hodnotit pouze z morfologického hlediska. Vzhledem k
charakteru tvorby příznaků z biosignálu lze také předpokládat vysokou citlivost
na šum a artefakty (viz sekce~\ref{sec:hybridni_detekce}). Je třeba dále
poznamenat, že se jedná o pouhé \enquote{beta} testy a modely, vzhledem k tomu,
že nebyla věnována pozornost širší optimalizaci. I přes tento fakt bylo ve
srovnání se současným stavem dosaženo velmi dobrých výsledků v rámci
benchmarkovacích datasetů, což naznačuje, že navržené řešení je efektivnější a
flexibilnější. Srovnání je uvedeno v Tabulce~\ref{tab:state_comparison}.

Velka část současných řešení využívá přístupy strojového učení založené na učení
s učitelem nebo hlubokém učení (viz
kapitola~\ref{sec:machine_learning})~\cite{Ishaque2021}. V případě technik učení
s učitelem lze hovořit o takzvaných \enquote{data-driven} metodách, protože k
hodnocení kognitivní zátěže využívají \gls{HRV} parametry. To přináší omezení
například v podobě délky časového úseku, které parametry potřebují pro svůj
výpočet. Zároveň všechny tyto přístupy nezohledňují příčinné souvislosti mezi
použitými příznaky. To činí interpretaci výsledků složitou, a proto jsou
přístupy poměrně omezené při dalším zkoumání kauzálních vztahů mezi příznaky.
Obvykle také tyto modely využívají po natrénování desítky milionů extrahovaných
parametrů pro účely predikce. Realizované řešení se díky kapsulární neuronové
síti dostává na řád tisíců. Přichází tak v úvahu i myšlenka embedded aplikací.
Navržená metodika je detailně rozebrána v kapitole~\ref{sec:hybridni_detekce}.
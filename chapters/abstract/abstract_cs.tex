Diplomová práce se věnuje hodnocení kognitivní zátěže v extrémních prostředích,
což je kritické pro úspěch a bezpečnost jednotlivců a týmů při náročných a
důležitých úkonech. Tradiční metody monitorování pomocí dotazníků nebo
behaviorální analýzy mohou být v extrémních podmínkách nepraktické až
neproveditelné. Z tohoto důvodu stále více roste zájem o využití periferních
biosignálů pro hodnocení kognitivní zátěže v reálném čase. Práce konkrétně
zkoumá vliv extrémního prostředí v podobě vesmírné analogové mise na projevy
kognitivní zátěže v elektrické srdeční, respirační a elektrodermální aktivitě.
Pro účely hodnocení kognitivní zátěže je představen nový multimodální způsob
založený na tvorbě fyziologických příznaků ve formě vícerozměrných
časoprostorových kauzálních vzorů, které umožňují unikátní kódování specifického
kognitivního stavu. Kapsulární neuronová sít je navržena pro synergické
sjednocení vytvořených fyziologických příznaků využitím autoenkodérové komprese
do jednotného latentního prostoru k zachycení časoprostorových kauzálních
relací. Navržené řešení je otestováno na populárních veřejně dostupných
benchmarkovacích datasetech včetně dat z analogové vesmírné mise.


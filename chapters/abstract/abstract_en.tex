The thesis focuses on the assessment of cognitive load in extreme environments,
which is critical for the success and safety of individuals and teams performing
demanding and essential tasks. Traditional monitoring methods using
questionnaires or behavioral analysis may be impractical or even impossible in
extreme conditions. For this reason, there is a growing interest in using
peripheral biosignals for real-time cognitive load assessment. Specifically, the
thesis examines the impact of extreme environments, such as an analog space
mission, on the manifestations of cognitive load in electrical cardiac,
respiratory, and electrodermal activity. To assess the cognitive load, a new
multimodal approach is introduced based on the creation of physiological
features in the form of multivariate spatiotemporal causal patterns, allowing
for a unique encoding of specific cognitive states. A capsular neural network is
designed for synergic uniform integration of the physiological features to
capture spatiotemporal causal relations by exploiting autoencoder compression
capability. The proposed solution is tested on popular publicly available
benchmark datasets, including data from an analog space mission.
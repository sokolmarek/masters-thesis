Lidský mozek je schopen se neustále přizpůsobovat měnícím se okolnostem a
požadavkům prostředí. Astronauti se musí aklimatizovat na zcela nové prostředí
podobně jako malé děti procházejí svými vývojovými fázemi. V podmínkách stavu
beztíže nebo mikrogravitace dochází mimo jiné k ovlivnění mozkových procesů.
Cílem kognitivních neurověd ve vesmíru je pochopit, jak mozek a mysl reagují na
tyto jedinečné okolní podmínky. První výzkumy v oblasti neurověd ve vesmíru byly
provedeny v roce 1962 během ruské mise Vostok-3. Na Zemi je oproti vesmíru možné
díky neurozobrazovacím technikám snadno studovat mozkovou aktivitu a kognitivní
funkce. Pro neurovědce i psychology je velmi důležité pochopit základní
neurokognitivní a neuropsychologické aspekty kosmického letu. Neefektivní
mentální výkon jedince či posádky je známou hrozbou pro vesmírné mise. Pozemský
výzkum zdůrazňuje nutnost porozumět kognitivním procesům, jelikož \gls{CL}
zhoršuje kognitivní a percepční motorické schopnosti. Srovnatelné dopady lze
předpokládat i při vesmírných misích v náročných \gls{ICE} podmínkách a
simulacích (analogové mise)~\cite{Torre2014}. Pro potřeby diplomové práce je
rozsáhlá kapitola kognitivních neurověd vymezena popisu kognitivní zátěže.

\subsection{Terminologie}
\label{subsection:terminologie_CL}
Obor kognitivních neurověd se nachází na pomezí psychologie a neurověd, ale
překrývá se s dalšími obory jako kognitivní a biologická psychologie, fyziologie
nebo neuropsychologie. Tyto obory často studují podobné procesy ale z jiného
pohledu a pomocí jiné metodiky. Synonymem oblasti kognitivních neurověd byl v
této práci zvolen termín neuropsychofyziologie pro usnadnění popisu
fyziologických změn v důsledku kognitivních procesů. Cílem této práce není
rozlišovat mezi jednotlivými obory ale zaměřit se na vědecké otázky z jejich
průniků, které jsou spojené s vlivem \gls{ICE} prostředí na člověka.

\subsection{Kognitivní zátěž}
\label{subsection:kognitivni_zatez}
Prvotní pojetí kognitivní zátěže (\gls{CL}, Cognitive Load) pochází z oblasti
výuky a vzdělávání, kterou se intenzivně zabýval Sweller et
al.~\cite{Sweller1988,Sweller1998,Sweller2010} kolem přelomu 19. a 20. století.
Sweller~\cite{Sweller1988} jako první formuloval teorii kognitivní zátěže
(\gls{CLT}, Cognitive Load Theory), podle které je \gls{CL} definována jako
zvýšené požadavky na ukládání a zpracování informací v pracovní paměti člověka.
Chen et al.~\cite{Chen2016} definoval kognitivní zátěž jako proměnnou, jež
určuje míru požadavků kladených úkolem na dostupné mentální zdroje pro
zpracování informací. Míra požadavků vychází z vnímané námahy při učení, myšlení
a uvažování jakožto ukazatel zatížení pracovní paměti během plnění úkolu. Tato
míra zároveň popisuje interakce mezi nároky na zpracování úkolu a lidskou
mentální výkonností~\cite{Haapalainen2010}. Ačkoli se definice kognitivní zátěže
v jednotlivých oborech mohou lišit, všechny mají jeden základní prvek: procento
využité kapacity lidské pracovní paměti. Na rozdíl od naší smyslové a dlouhodobé
paměti, které mohou v podstatě neomezeně zpracovávat informace, je tato kapacita
omezená~\cite{Vanneste2021}.

Chen et al.~\cite{Chen2016} publikoval, že vysoká kognitivní zátěž (kognitivní
přetížení) může mít negativní dopad na výkonnost pracovní paměti. Vliv
kognitivní zátěže zároveň ovlivňuje mozkové procesy a kognitivní přetížení
pravděpodobně bezprostředně předchází vyhoření~\cite{Vanneste2021}. Většina
kognitivních funkcí, jako je například selektivní pozornost nebo sebekontrola,
závisí na pracovní paměti. Pracovní paměť zahrnuje aktivní krátkodobé ukládání,
zpracování a manipulaci s informacemi a její efektivní fungování je kriticky
závislé na inhibičních nervových procesech. Nervové středisko pracovní paměti
sestává hlavně z distribuované sítě struktur zahrnující prefrontální kortex jako
důležité ohnisko. Variabilita srdeční frekvence (\gls{HRV}) souvisí s aktivitou
prefrontální kůry a je inverzně spojena s aktivitou subkortikálních struktur
jako je amygdala~\cite{Thayer2009}. Vliv kognitivní zátěže se také promítá do
elektrodermální (\gls{EDA}) a respirační (\gls{RSP})
aktivity~\cite{Mogilever2018}. Díky periferním biologickým signálům se tedy
naskytuje jednoduší možnost hodnocení či predikce kognitivních procesů bez
nutnosti použití realizačně a finančně náročnějších neurozobrazovacích metod
nebo elektroencefalografie (\gls{EEG}). Detailněji jsou fyziologické změny
spojené s vlivem \gls{CL} popsány v následující sekci.

\subsection{Fyziologické projevy}
\label{subsec:fyziologicke_projevy_CL}
Z předchozího textu již může být zřejmé, že kognitivní zátěž je velmi úzce
spojená s fyziologickými změnami v lidském organismu. U jedince, kde dojde k
stimulaci CL, může docházet k řadě změn, například v mozkové aktivitě, krevním
tlaku, srdečním rytmu, rychlosti pulzní vlny, dýchaní nebo činnosti potních
žláz~\cite{Vanneste2021,Haapalainen2010,Thayer2009,Gjoreski2017,Cruz2019,Brouwer2015}.
Vybrané fyziologické funkce, do kterých se po promítá kognitivní zátěž, jsou
společně s jejich krátkým popisem a neurobiologickým vztahem uvedeny v
Tabulce~\ref{tab:prehled_fyziologicke_projevy_CL_tab1}. Tabulka zároveň
poukazuje na rozdílné vztahy fyziologických parametrů s kognitivní zátěží.

\begin{table}[ht]
    % \setlength{\tabcolsep}{10pt}
    \renewcommand{\arraystretch}{1.5}
    \centering
    \begin{threeparttable}
        \caption{Přehled vybraných nejčastěji studovaných fyziologických změn
            a stručné teoretické zdůvodnění jejich souvislosti s kognitivní zátěží
            (Upraveno a převzato z~\cite{Vanneste2021})}
        \label{tab:prehled_fyziologicke_projevy_CL_tab1}
        \scriptsize
        \begin{tabular}{p{3cm}p{11cm}}
            \toprule
            Fyziologická funkce & Stručný popis a teoretické zdůvodnění předpokládaného vztahu mezi fyziologickým měřením a kognitivní zátěží
            \\ \midrule
            EEG                 & \textit{Popis}: \gls{EEG} umožňuje měřit mozkovou aktivitu neinvazivně. Provedení spektrální analýzy naměřených rozdílů elektrických potenciálů umožňuje analyzovat výkon různých frekvenčních pásem, která jsou v signálu\newline
            \rule{0pt}{2.5ex}\noindent
            \textit{Hypotéza}: Zvýšení kognitivní zátěže lze měřit zvýšením mozkové aktivity, tj. oscilací v určitém frekvenčním pásmu s větší amplitudou~\cite{Antonenko2010}
            \\
            Eye-tracking        & \textit{Popis}: Měření průměru zornice, latence mrknutí a charakteristik sakád\newline
            \rule{0pt}{2.5ex}\noindent
            \textit{Hypotéza}: Sledování očí bylo v předchozích studiích spojeno s kognitivní zátěží prostřednictvím neurobiologických mechanismů, jako je inervace neuronů autonomního nervového systému radiálními vlákny duhovky~\cite{Wel2018}
            \\
            EDA                 & \textit{Popis}: Elektrodermální aktivita, hodnotí elektrické charakteristiky kůže, aby bylo možné odvodit změny vlivem sympatického nervového systému.\newline
            \rule{0pt}{2.5ex}\noindent
            \textit{Hypotéza}: Kognitivní zátěž má vliv na stres nebo vzrušení a vede ke zvýšení kožní vodivosti~\cite{Setz2010}
            \\
            Teplota pokožky     & \textit{Popis}: Měření teploty vnějšího povrchu lidského těla\newline
            \rule{0pt}{2.5ex}\noindent
            \textit{Hypotéza}: Kognitivní zátěž má vliv na stres nebo vzrušení a vede k vazokonstrikci, což snižuje teplotu kůže~\cite{Herborn2015}
            \\
            EKG                 & \textit{Popis}: Srdeční frekvenci a variabilitu srdeční frekvence lze hodnotit pomocí elektrokardiogramu nebo fotopletysmografie\newline
            \rule{0pt}{2.5ex}\noindent
            \textit{Hypotéza}: Tepová frekvence je měřítkem aktivity sympatického i parasympatického autonomního nervového systému. Stres nebo vzrušení způsobí zvýšení krevního tlaku a snížení variability srdeční frekvence~\cite{Jercic2020,Solhjoo2019}
            \\
            RSP                 & \textit{Popis}: Kognitivní zátěž vede ke zrychlenému dýchání a vyšší minutové ventilaci, přičemž dechová amplituda zůstává stejná\newline
            \rule{0pt}{2.5ex}\noindent
            \textit{Hypotéza}: Při soustředění pozornosti nebo plnění náročného úkolu dochází ke změnám v dýchání~\cite{Grassmann2016}
            \\
            \bottomrule
        \end{tabular}
    \end{threeparttable}
\end{table}

Příkladem rozdílností může být EEG, u kterého je vztah s kognitivní zátěží
přímější než u srdeční nebo elektrodermální aktivity. Hodnocení pomocí tohoto
biosignálu je tedy v některých případech spolehlivější. Pouze v některých,
protože se vychází z předpokladu, že veškeré změny v kognitivních funkcích
člověka se odrážejí v jeho fyziologii~\cite{Vanneste2021}. To znamená, že žádná
měřicí technika nemůže sama o sobě zachytit všechny aspekty kognitivních funkcí
nebo o nich jednoznačně vypovídat. Problematika těchto vztahů je nadále
rozebírána v kapitolách~\ref{subsec:detekce_CL}
a~\ref{sec:neurovisceralni_integrace}. Na tuto vícerozměrnou povahu kognitivní
zátěže poukázal Kramer et al.~\cite{Kramer1991}. Dále definoval kritéria:
citlivost, diagnostičnost, rušivost, spolehlivost a obecnost použití, v rámci
kterých bude mít každé fyziologické měření jinou povahu. V případě kognitivní
zátěže a její spolehlivém hodnocení z fyziologických projevů je proto vhodný
multimodální\footnote{Multimodálním přístupem je myšleno využití více
    biosignálů} přístup, který může poskytnout její robustnější
reprezentaci~\cite{Chen2016}.

\subsection{Detekce kognitivní zátěže}
\label{subsec:detekce_CL}
Detekce kognitivní zátěže pomocí periferních biosignálů patří mezi tři
nejčastěji používané způsoby včetně subjektivního hodnocení a hodnocení založené
na výkonnosti jedince. Subjektivní hodnocení vychází z předpokladu, že hodnocený
subjekt je schopen vnímat své vlastní kognitivní procesy a informovat o případně
kognitivní zátěží nebo o množství vynaloženého mentálního
úsilí~\cite{Wang2019,Schnotz2007}. K subjektivnímu hodnocení se dále využívají
specifické dotazníky, mezi které patří například NASA-TLX~\cite{Schnotz2007}.
Jedná se o vícerozměrnou stupnici používanou k měření pracovní zátěže operátorů
v rámci úkonů například během vesmírných analogových misí~\cite{Sandra2006}.
Tyto metody ale nejsou předmětem této práce a již byly detailně popsány
v~\cite{Schnotz2007}.

\begin{figure}[!htb]
    \begin{center}
        \includegraphics[width=0.75\linewidth]{figures/physiological_measures}
        \caption{Běžné fyziologické a fyzické měření související s \gls{CL} (Přeloženo
            a převzato z~\cite{Giannakakis2022})}
        \label{fig:physiological_measures}
    \end{center}
\end{figure}

Na kognitivní zátěž lze nahlížet jako na multidimenzionální konstrukt, který
reprezentuje zátěž vyvinutou na jedince~\cite{Wang2019}. Zároveň se efekt
stimulace kognitivních funkcí u každého jedince projevuje jinak. Nelze tedy
vytvořit žádnou univerzální metodu, která by byla schopná stejně spolehlivě
detekovat kognitivní zátěž u různých subjektů. Proto se v tomto odvětví nabízejí
a velmi často využívají metody strojového učení v rámci kterých nachází výhodné
uplatnění dříve zmíněný multimodální přístup. Biologická data využívaná pro
účely detekce \gls{CL} lze vidět na Obr.~\ref{fig:physiological_measures}.
Nejčastěji využívaným biosignálem je srdeční aktivita, konkrétně variabilita
srdeční frekvence, které se často používá v kombinaci s
elektrodermální aktivitou~\cite{Wang2019}.

Za konvenční metodu detekce \gls{CL} by se dalo považovat jednoduché porovnávání
trendů parametrů s baseline, vypočtených například z variability srdečního
rytmu. Z variability srdeční frekvence je však v současnosti možné vypočítat
desítky parametrů, se kterými je v těchto úlohách nutné velmi opatrně zacházet.
Jednotlivé \gls{HRV} parametry společně s jejich pravděpodobným významem byly
již popsány v~\cite{Haapalainen2010,Rohila2020,Pham2021,Bouny2021}. Blíže je
problematika \gls{HRV} parametrů popsána v samotné
podkapitole~\ref{subsec:hrv_indices}.

\begin{table}[h]
    \centering
    \caption{Veřejně dostupné datasety sledující kognitivní zátěž (Upraveno a
        převzato z~\cite{Gjoreski2020})}
    \label{tab:cl_datasets}
    \scriptsize
    \begin{tabular}{lcp{9cm}}
        \toprule
        \textbf{Dataset}      & \textbf{Probandi} & \textbf{Biosignály}                                                          \\ \midrule
        Ascertain             & 58                & ECG, EDA, EEG, aktivační jednotky obličeje                                   \\
        Amigos                & 40                & EEG, ECG, GSR, video obličeje                                                \\
        DEAP                  & 32                & ECG, EDA, EEG, EMG, EOG, RSP, TEMP, video obličeje                           \\
        DECAF-hudba           & 30                & ECG, EMG, EOG, MEG, near-infrared video obličeje                             \\
        DECAF-video           & 30                & ECG, EMG, EOG, MEG, near-infrared video obličeje                             \\
        Mahnob                & 30                & ECG, EDA EEG, RSP, TEMP, video, sledování očí, zvuk                          \\
        Emotions              & 1                 & ECG, EDA, EMG, RSP                                                           \\
        Laughter              & 34                & ACC, EDA, PPG, TEMP                                                          \\ \midrule
        Pracovní zátěž řidiče & 10                & GSR, HR, TEMP                                                                \\
        Řidičský stres        & 24                & ECG, EDA, EMG, RESP                                                          \\
        Rozptýlení řidiče     & 64                & GSR, HR, TEMP ECG, EDA, EMG, RSP EDA,HR, RSP, výrazy obličeje, sledování očí \\ \midrule
        CLAS                  & 59                & ACC, ECG, PPG, EDA                                                           \\
        Stress-math           & 21                & ACC, EDA, HR, TEMP, BVP ACC, EDA, HR, TEMP, BVP                              \\
        Non-EEG               & 20                & ACC, EDA, HR, TEMP, BVP ACC, EDA, HR, TEMP, SpO2                             \\
        WESAD                 & 15                & ACC, EDA, TEMP, BVP, EMG, RSP                                                \\ \midrule
        CogLoad               & 23                & ACC, EDA, TEMP, RR                                                           \\
        Snake                 & 23                & ACC, EDA, TEMP, RR                                                           \\ \bottomrule
    \end{tabular}
\end{table}




To this end, we train binary classifiers based on Logistic Regression, (Boosted)
Decision Trees, Random Forests, and Support Vector Machines. We find that, while
ML models trained directly on the raw sensor measurements are better than random
guessing, there are substantial further accuracy improvements to be gained
through feature engineering.

In ML, substantial improvements in accuracy are often achieved by preprocessing
raw feature vectors before model training.

feature transformation, feature extraction, and feature selection techniques

Cognitive load (CL), which is originally a psychological term to describe how
much working memory resources are used, is used as a metric to enable the design
of intelligent devices that adapt their interactions with the user. To this end,
it is important for the device to be aware of the user’s current CL.

These studies also explored a variety of machine learning (ML) methods including
K-nearest neighbors (KNN), artificial neural net­ work (ANN), support vector
machine (SVM), naïve Bayes (NB), decision tree (DT), linear discriminant
analysis (LDA), and logistic regression (see Chapter 4 for an overview of these
methods).